\documentclass[../main.tex]{subfiles}

%{definition}{Definice}
%{example}{Příklad}
%{intuition}{Intuice}
%{remark}{Poznámka}
%{consequence}{Důsledek}
%{observation}{Pozorování}

\begin{document}
%%%%%%%%%%%%%%%%%%%%%%%%%%%%%%%%%%%%%%%%%%%%%%%%%%%%%%%%%%%%%%%%%%%%%%%%%%%%%%%%
\section{První přednáška}

Využití:
\begin{enumerate}
    \item počítačová grafika, animace
    \item počítačový design
    \item robotika, mechanika, CNC stroje
    \item zpracování obrazu, umělé vidění...
\end{enumerate}

\subsection{Shodná zobrazení}

Existuje šest shodností v rovině:
\begin{enumerate}
    \item osová souměrnost
    \item otočení
    \item středová souměrnost
    \item posunutí
    \item posunuté zrcadlení
    \item identita
\end{enumerate}

\begin{definition}
    osová souměrnost\\
    Nechť je daná přímka $o$, kterou nazýváme osa souměrnosti. Potom pro 
    obraz $M'$ libovolného bodu M této přímky $o$ platí $M' = M$. Ke každému
    $X$, který neleží na přímce $o$ sestrojíme obraz $X'$ následujícím způsobem:\\

    Bodem $X$ vedeme kolmici $k$ na přímku $o$ a její patu označíme $X_0$. Na 
    polopřímce opačné k polopřímce	 $X_0 X$ sestrojíme bod $X'$ tak, že $|X'X_0|=|XX_0|$.
    Takto definované zobrazení nazýváme osová souměrnost s osou $o$.
    \begin{example}
        Je daná přímka $p$ a body $A,B$ v téže polorovině s hraniční přímkou $p$.\\
        Najděte všechny body $X\in p$ takové, aby součet $|AX|+|BX|$ byl minimální.\\
        Řešení: jeden z bodů promítneme pomocí souměrnosti na druhou stranu a spojíme s druhým.\\
        Pak bod $X$ bude průsečník $|AB'|$ a $p$.
        \\\textbf{zo slidov 4. slide}
    \end{example}
\end{definition}

\begin{remark}

    \begin{enumerate}
        \item Shodné zobrazení, jehož všechny samodružné body vyplní přímku $o$, je souměrnost
        podle osy $o$ (alternativní definice).
        \item Jestliže existují na přímce dva různé samodružné body, pak každý bod této přímky
        je samodružný.
        \item Má-li shodnost alespoň tři nekolineární samodružné body, je to identita.
        \item Má-li shodnost dva různé samodružné body a není identitou, pak je osovou souměrností.
        \item Samodružné přímky osové souměrnosti jsou přímky kolmé na osu souměrnosti.
    \end{enumerate}
\end{remark}

\begin{example}
    Napište analytické vyjádření osové souměrnosti s osami $x$ a $y$.\\
    Řešení: Vezmeme na ose $x$ bod $M=M'$. Víme, že všechny body na $x$ budou samodružné.
    Jiné body, které neleží na přímce (osi) $x$ bude mít nějaké souřadnice $B = [x,y]$.
    Potom $B' = [x,-y] = [x',y'] \forall x,y$. Víme, že $x' = y$ a $y' = -y$. To je analytické vyjádření
    osové souměrnosti podle osi $x$. Podobně obráceně pro $y$. 
\end{example}

\begin{example}
    Obecný tvar přímky: Napište analytické vyjádření osové souměrnosti podle osy \\
    $o:ax+by+c=0$ (Potažmo konkrétně 3x+4y-7=0).\\
    Řešení:(\textbf{screenshot z prednášky, okolo 50. minuty záznamu})
\end{example}

\begin{definition}
    Otočení\\
    Otočení (rotace) je zobrazení určené středem S a orientovaným úhlem velikosti $\phi$, které bodu
    S přirazuje týž bod S a libovolnému bodu $X \neq S$ přiřazuje bod $X'$ tak, že $|X'S| = |XS|$ a orientovaný
    úhel $XSX'$ má velikost $\phi$. Bod S nazýváme střed otáčení a $\phi$ je úhel otáčení.
\end{definition}
\begin{remark}

    \begin{enumerate}
        \item Shodnost, která není ani identitou ani osovou souměrností, má nejvýše jeden samodružný bod.
        \item Shodnost s právě jedním samodružným bodem je rotace (alternativní definice).
        \item Složením dvou osových souměrností s růžnoběžnými osami vznikne otočení, jehož středem je průsečník těchto os.
        \item Otočení se středem S a úhlem $\alpha$ převádí přímku $p$ v přímku $p'$ různoběžnou s $p$. Přitom
        dva vrcholové úhly, které $p$ a $p'$ svírají, mají velikost $\alpha$.
        \item Složením posunusí, rotace $R(O,\alpha)$ a posunutím dostaneme rotaci o libovolném středu.
    \end{enumerate}
\end{remark}
\end{document}