\documentclass[../main.tex]{subfiles}

\begin{document}

\section{První přednáška}

\begin{definition}[Metrický prostor]
    Metrický prosto je dvojice $(M,d)$ množiny $M\neq \emptyset$ a
    zobrazení $d: M \times M \to \mathbb{R}$ zvaného metrika či vzdálenost, které
    $\forall x,y,z \in M$ splňuje:
    \begin{enumerate}
        \item
        $d(x,y) = 0 \Leftrightarrow x=y$
        \item
        $d(x,y) = d(y,x)$
        \item
        $d(x,y)\leq d(x,z) + d(z,y)$
    \end{enumerate}

    \noindent
    Z této definice plyne i nezápornost.
\end{definition}

\begin{definition}[Podprostor]
    Každá podmnožina $X \subset M$ určuje nový metrický prostor $(x,d')$, tak zvaný
    podprostor metrického prostoru $(M,d)$: pro $x,y \in X$ klademe
    $d'(x,y) := d(x,y)$. Obě metriky označíme stejným symbolem a máme $(X,d)$.
\end{definition}

\begin{definition}[Izometrie $f$]
    Izometrie $f$ dvou metrických prostorů $(M,d)$ a $(N,e)$ je bijekce
    $f:M\to N$, jež zachovává vzdálenosti:
    \[ \forall x,y \in M: d(x,y) = e(f(x), f(y)) \]
    Pokud existuje, prostory $(M,d)$ a $(N,e)$ jsou \underline{izometrické}.
\end{definition}

\begin{example}[Euklidovský prostor $(\mathbb{R}^n, e_n)$]
    Jedním z nejdůležitějších příkladů metrických prostorů je ($n$-rozměrný)
    Euklidovský prostor $(\mathbb{R}^n, e_n), n \in \mathbb{N}$ s metrikou
    $e_n$ danou pro
    $\overline{x} = (x_1, ..., x_n), \overline{y} = (y_1, ... y_n) \in \mathbb{R}^n$
    formulí \[ e_n(\overline{x}, \overline{y}) := \sqrt{\sum^n_{i=1}(x_i-y_i)^2}. \]
    Geometricky je $e_n$ délka úsečky určené body $\overline{x}$ a $\overline{y}$.
    Euklidovským prostorem pak rozumíme obecněji každý podprostor $(x,e_n)$, když
    $X \subset \mathbb{R}^n$.
\end{example}

\begin{lemma}[$(\mathbb{R}^n, e_n)$ je MP.]
    $(\mathbb{R}^n, e_n)$ je Metrický prostor.
\end{lemma}
\begin{proof}
    TODO
\end{proof}

\begin{example}[Sférická metrika]
    Jako \[ S := \{ (x_1,x_2,x_3) \in \mathbb{R}^3 | x_1^2 + x_2^2 + x_3^2 = 1 \} \]
    označíme jednotkovou sféru (s poloměrem 1) v Euklidovském prostoru $\mathbb{R}^3$.
    Funkci $s: S \times S \to [0,\pi]$ definujeme pro $\overline{x}, \overline{y} \in S$ jako
    \[ s(\overline{x}, \overline{y}) =
    \begin{cases} 0 \hspace{2mm} ... \hspace{2mm} \overline{x} = \overline{y}\\ 
        \varphi \hspace{2mm} ... \hspace{2mm} \overline{x} \neq \overline{y}
    \end{cases} \]
    kde $\varphi$ je úhel sevřený dvěma polopřímkami vycházejícími z počátku $\overline{0} := (0,0,0)$
    a body $\overline{x}$ a $\overline{y}$. Tento úhel je vlastně délka kratšího z oblouků mezi body
    $\overline{x}$ a $\overline{y}$ na jednotkové kružnici vytknuté na $S$ rovinou určenou počátkem
    a body $\overline{x}$ a $\overline{y}$. Funkci nazveme sférickou metrikou.
\end{example}

\begin{lemma}[$S$ je Metrický Prostor]
    $(S,s)$ je metrický prostor.
\end{lemma}
\begin{proof}
    TODO
\end{proof}

\begin{definition}[(Horní) hemisféra $H$]
    (Horní) hemisféra $H$ je množina
    \[ H := \{ (x_1, x_2, x_3) \in S | x_3 \geq 0 \} \subset S. \]
\end{definition}

\begin{theorem}[$H$ není plochá]
    Metrický prostor $(H,s)$ není izometrický žádnému Euklidovskému prostoru
    $(X,e_n)$ s $X\subset \mathbb{R}^n$
\end{theorem}
\begin{proof}
    TODO
\end{proof}

\begin{definition}[Ultrametrika]
    Metrika $d$ v metrickém prostoru $(M,d)$ je ultrametrika nebo také nearchimédovská metrika, pokud
    splňuje silnou trojúhélníkovou nerovnost
    \[ \forall x,y,z \in M: d(x,y) \leq \max(d(x,z), d(z,y)). \]
    Protože $\max(d(x,z),d(z,y)) \leq d(x,z) + d(z,y)$, je každá ultrametrika metrika.
\end{definition}

\begin{remark}
    V ultrametrických prostorech, krátce UMP, neplatí intuice založená na Euklidovských prostorech.
\end{remark}

\begin{lemma}[trojúhelníky v UMP]
    V ultrametrickém prostoru $(M,d)$ je každý trojúhélník rovnoramenný (má dvě stejně dlouhé strany).
\end{lemma}
\begin{proof}
    TODO
\end{proof}

\begin{definition}[Koule]
    (Otevřená) koule se středem $a \in M$ a poloměrem $r>0$ b metrickém prostoru $(M,d)$
    je podmnožina
    \[ B(a,r) := \{ x \in M | d(x,a) < r \} \subset M. \]
    Vždy platí, že $B(a,r) \neq \emptyset$, protože $a \in B(a,r)$.
\end{definition}

\begin{example}[$p$-adické metriky]
    Nechť $p \in \{ 2,3,5,7,11,\dots \}$ je prvočíslo a nechť $n \in \mathbb{Z}$
    je nenulové celé číslo. Definujeme $p$-adický řád čísla $n$:
    \[ \textrm{ord}_p(n) := \max(\{ m \in \mathbb{N}_0 : p^m | n \})\text{\footnotemark} \]
    \footnotetext{Zde $\cdot | \cdot$ značí relaci dělitelnosti na $\mathbb{Z}$, kde $a,b\in \mathbb{Z}$
    je $a|b \Leftrightarrow \exists c \in \mathbb{Z}: b = ac$.}
    Pro každé $p$ definujeme $\textrm{ord}_p(0) := +\infty$.

    Funkci $\textrm{ord}_p(\cdot)$ rozšíříme na zlomky.
    Pro nenulové $\alpha = \frac{a}{b} \in \mathbb{Q}$ definujeme
    \[ \text{ord}_p(\alpha) := \text{ord}_p(a) - \text{ord}_p(a) \]
    a jinak znovu definujeme $\text{ord}_p(0) = \text{ord}_p(\frac{0}{b}) := +\infty$.
\end{example}

\begin{lemma}[Aditivita $\textrm{ord}_p(\cdot)$]
    Platí, že \[ \forall \alpha ,\beta \in \mathbb{Q}:
    \text{ord}_p(\alpha \beta) = \text{ord}_p(\alpha) + \text{ord}_p(\beta), \]
    kde $(+\infty) + (+\infty) = (+\infty) + n = n + (+\infty) := +\infty$ pro každé $n \in \mathbb{Z}$.
\end{lemma}
\begin{proof}
    TODO
\end{proof}

\begin{definition}[$p$-adické normy]
    Fixujeme reálnou konstantu $c \in (0,1)$ a definujeme funkci\\
    $| \cdot |_p: \mathbb{Q} \to [0, +\infty)$, tzv. $p$-adickou normu, jako
    \[ \left| \frac{a}{b} \right|_p := c^{\text{ord}_p(\frac{a}{b})}, \]
    speciálně $|0|_p = c^{+\infty := 0}$
\end{definition}

\begin{definition}[Normované tělěso $F$]
    Normované těleso $F = (F, 0_F, 1_F, +_F, \cdot _F, |\cdot|_F)$, psáno zkráceně
    $(F, |\cdot|_F)$, je těleso $F$ vybavené normou $|\cdot|_F: F \to [0, +\infty)$, jež
    splňuje tři následující požadavky:
    \begin{enumerate}
        \item $ \forall x \in F: |x|_F = 0 \Leftrightarrow x = 0_F $
        \item $ \forall x,y \in F: |x\cdot _F y|_F = |x|_F \cdot |y|_F $
        \item $ \forall x,y \in F: |x +_F y|_F \leq |x|_F + |y|_F $
    \end{enumerate}
\end{definition}

\begin{example}
    Základní příklady normovaných těles jsou např. $\mathbb{Q}, \mathbb{R}$ nebo $\mathbb{C}$, kde
    je normou obvyklá absolutní hodnota $|\cdot|$
\end{example}

\begin{lemma}[o $|\cdot|_p$]
    Pro každé prvočíslo $p$ a každé $c \in (0,1)$ je $(\mathbb{Q}, |\cdot|_p)$ normované těleso.
    Příslušný metrický prostor $(\mathbb{Q}, d)$, je ultrametrický prostor.
\end{lemma}
\begin{proof}
    TODO
\end{proof}

\vspace{3mm}
\noindent
\textbf{Úlohy 1, 3, 5, 7 a 19 jsou DÚ(slajdy) do 9.3.}

\end{document}