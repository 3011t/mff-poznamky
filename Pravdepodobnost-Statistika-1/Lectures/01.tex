\documentclass[../main.tex]{subfiles}

%{definition}{Definice}
%{example}{Příklad}
%{intuition}{Intuice}
%{remark}{Poznámka}
%{consequence}{Důsledek}
%{observation}{Pozorování}

\begin{document}
%%%%%%%%%%%%%%%%%%%%%%%%%%%%%%%%%%%%%%%%%%%%%%%%%%%%%%%%%%%%%%%%%%%%%%%%%%%%%%%%
\section{První přednáška}

\subsection{Úvodem}

Modely náhody $\rightarrow$ Pravděpodobnost $\rightarrow$ Pozorovaná data 
$\rightarrow$ Modely náhody\\
Model náhody např. kostka 1,\dots,6,\\
Pozorovaná data : 1,5,4,3,3\\
otázka na pravděpodobnost: jaká je pravděpodobnost\dots
hodně pozorovaných dat $\rightarrow$ statistika na model náhody.

\begin{example}[Schwartz-Zippel algoritmus]
    Máme dány dva polynomy $f(x),g(x)$ stupně $d$. Chceme zjistit, zda jsou stejné, 
    a to co nejrychleji.

    \noindent
    Problém: $g(x)$ je součin několik polynomů stupně $\leq \frac{d}{4}$, 
    dostávame víc než lineární čas.\\
\end{example}

\begin{solution}
    Algoritmus: zvolíme náhodně $x \in \{1,2\dots,100d\}$, ověříme, 
    zda $f(x_1) = g(x_1)$. Když $f \neq g$, tak $x_1$ je kořen polynomu $f-g$.
    \dots takových $x_1$ je  $\leq d$.\\
    \[P(f(x_1)=g(x_1) : f\neq g) \leq \frac{1}{100}\]
    Pokud jsme spokojeni s 1\%, končíme, když ne, volíme
    $x_2,x_3... \in \{1,2,\dots,100d\}$, pak 
    \[P(Pro x_1,x_2,x_3\dots  f(x_i)=g(x_i) : f\neq g) \leq 
    \left(\frac{1}{3}\right)^3 = 10^{-6}\]
    ... aproximační algoritmy
\end{solution}

\vspace{5mm}
\noindent 
Některé jevy neumíme/nechceme popsat kauzálně
\begin{itemize}
    \item hod kostkou
    \item tři hody kostkou, nekonečně mnoho hodů kostkou
    \item hod šipkou na terč
    \item počet emailů za den
    \item dobu běhu programu (v reálnem počítači)
    \item a další ...
\end{itemize}
Důvody:
\begin{itemize}
    \item fyzikální vlastnost přírody
    \item komplikovaný proces (počasí, medicína, molekuly plynu...)
    \item neznáme vlivy (působení dalších lidí, programů...)
    \item randomizované algoritmy (test prvočíselnosti, quicksort)
    \item náhodné grafy (Ramseyovy čísla)
    \item a další ...
\end{itemize}


\noindent
Pro popis pomocí teorie pravděpodobnosti napřed vybereme množinu elementárních jevů $\Omega$
(sample space)
\[\Omega = \{1,2\dots,6\} = [6] \implies \text{hod kostkou}\]
\[\Omega = [6]^3 \implies \text{hod třemi kostkami}\]

\subsection{Základní definice}
\begin{definition}[Prostor jevů]
    $\mathcal{F} \subseteq \mathcal{P}(\Omega)$

    $\mathbb{F} \subseteq \mathbb{P}(\Omega)$ je prostor jevů (též $\sigma$-algebra), pokud
    \begin{enumerate}
        \item $\emptyset \in \mathcal{F}$ a $\Omega \in \mathcal{F}$
        \item $A \in \mathcal{F} \implies \Omega \backslash A \in \mathcal{F}$
        \item $A_1,A_2,... \in \mathcal{F} \implies \bigcup_{i=1}^{\infty}A_i \in \mathcal{F}$
    \end{enumerate}
\end{definition}

\noindent
Často $\mathcal{F} = \mathcal{P}(\Omega)$, to je možné vždy, když je $\Omega$ spočetná,
např. pro $\Omega = \mathbb{R}$ to již nejde.

\begin{definition}[Pravděpodobnost]
    $P:\mathcal{F} \rightarrow [0,1]$ se nazývá pravděpodobnost (probability), pokud:
    \begin{enumerate}
        \item $P(\emptyset) = 0, P(\Omega) = 1$, a
        \item $P(\bigcup^\infty_{i=1}A_i) = \sum^\infty_{i=1}P(A_i)$, pro libovolnou posloupnost po dvou \textit{disjunktních} jevů \textbf{TODOOT}
    \end{enumerate}
    Šance (odds) jevu A je $O(A) = \frac{P(A)}{P(A^c)}$. Např. šance na výhru je 1 ku 2 znamená, že
    pravděpodobnost výhry je $\frac{1}{3}$; šance, že na kostce padne šestka je 1 ku 5.
\end{definition}

\noindent
\textbf{Konvence:}
\begin{itemize}
    \item
    \uv{A je jistý jev} znamená $P(A) = 1.$ Také se říká, že A nastáva skoro jistě (almost surely), zkráceně s.j. (a.s.).

    \item
    \uv{A je nemožný jev} znamená P(A) = 0.
    \[P(A) = 0 \Rightarrow^? A = \emptyset\]
    \[\leftarrow \text{axiom}\]
    \[\rightarrow \text{platí často, ne vždy}\]

    \item
    Např. $A = {\text{střed kruhu (házení šipek na terč})} \implies P(A) = 0$
    B spočetná (konečná, velká jako $\mathbb{N}$) množina:
    \[P(B) = 0+0+0+\dots=0\]
    $B_i$ je $i$-tý bod,
    $B = \bigcup B_i$
\end{itemize}

\begin{theorem}[Vlastnosti pravděpodobnostního prostoru]
    V pravděpodobnostním prostoru $(\Omega, \mathcal{F}, P)$ platí pro A,B $\in \mathcal{F}$:
    \begin{enumerate}
        \item $P(A) + P(A^c) = 1$
        \item $A \subseteq B \implies P(A) \leq P(B)$
        \item $P(A\cup B) = P(A) + P(B) - P(A\cap B)$
        \item $P(A_1\cup A_2\cup \dots) \leq \sum_i P(A_i)$ (subaditiva, Booleova nerovnost)
         (nevyžadujeme disjunktnost, pak by platila rovnost)
    \end{enumerate}
\end{theorem}
\begin{proof}
    \begin{enumerate}
        \item $\Omega = A\cup A^c$; $A,A^c$ disj.,\\
        $1 = P(\Omega) = P(A) + P(A^c)$
        \item $P(B) = P(A) + P(B\backslash A) \geq P(A)$
        \item Využíváme Princip Inkluze a Exkluze. (ale nevim jiste, jestli to staci)
        \item trik zdisjunktnění: z $A_1,A_2...$ uděláme disjunktní množiny
         \[B_1 = A_1, B_2 = A_2 \backslash A_1, B_3 = A_3 \backslash A_1 \cup A_2 \dots\]
         \[B_i \subseteq A_i \implies P(B_i) \leq P(A_i)\]
         \[B_i \cap B_j = \emptyset : j < i \dots B_i \cap B_j \subseteq B_i \cap A_j = \emptyset\]
         \[\bigcup^\infty_{i=1}B_i = \bigcup^\infty_{i=1}A_i\]
         \[\subseteq ok\]
         \[\text{opačná inkluze \textbf{TODOOT}}\]
    \end{enumerate}
    \[P(\bigcup A_i = P(\bigcup B_i) = \sum P(B_i) \leq \sum P(A_i)).\]
\end{proof}

\begin{example}[Pravděpodobnostní prostory]
    {\color{white} x} % Vkladame "bily znak" aby se enumerate spravne odradkovalo
    \begin{enumerate}
        \item 
        \textit{Konečný s uniformní pravděpodobností}

        $\Omega$ je libovolná konečná množina, $\mathcal{F} = \mathcal{P}(\Omega)$,\\
        $P(A) = \frac{|A|}{|\Omega|}$.
        \item 
        \textit{Diskrétní}

        $\Omega = \{\omega_1,\omega_2\dots\}$ je libovolná spočetná množina. Jsou dány $p_1,p_2\dots \in [0,1]$ se součtem 1.\\
        $P(A) = \sum_{i:\omega_i \in A} p_i$ (cinknutá loterie, nějaké možnosti mají jiné procenta)
        \item 
        \textit{Spojitý}

        $\Omega \subseteq \mathbb{R}^d$ pro vhodné $d$ ($\Omega$ např. uzavřená nebo otevřená)\\
        $\mathcal{F}$ vhodná (obsahuje např. všechny otevřené množiny)\\
        $f:\Omega \rightarrow [0,1]$ je funkcne taková, že $\int_{\Omega} f(x)dx = 1$.\\
        $P(A) = \int_{A}f(x)dx$

        Speciální případ: $f(x) = 1/V_d(\Omega)$\\
        $P(A) = \frac{V_d(A)}{V_d(\Omega)}$, kde $V_d(A) = \int_A 1$ je d-rozměrný objem A.
        
        \item 
        \textit{Bernoulliho krychle - nekonečné opakování}

        $\Omega = S^{\mathbb{N}}$, kde S je diskrétní s pravděpodobností $Q$,\\
        $\mathcal{F}$ vhodná (obsahuje např. všechny množiny tvaru\\
        $A = A_1 \times \dots \times A_k \times S \times S \times \dots $)\\
        $P(A) = Q(A_1) \dots Q(A_k)$\\
    \end{enumerate}
\end{example}

\newpage
\begin{example}[Nepříklady]
    {\color{white} x}
    \begin{enumerate}
        \item Náhodné přirozené číslo: můžeme si vybrat mnoha způsoby, Ale všechna přirozená čísla nemají stejnou pravděpodobnost.\\
        není možné, aby měly všechny stejnou nenulovou pravděpodobnost, protože pokud $P(0) = P(1) = P(2)\dots = P$ tak $P(\mathbb{N}) = p + p + p\dots = \infty$.
        \item Náhodné reálne číslo
        \item Betranův paradox
    \end{enumerate}
\end{example}

\begin{definition}[Podmíněná pravděpodobnost]

    Pokud $A,B \in \mathcal{F}$ a $P(B) > 0$, pak definujeme podmíněnou pravděpodobnost A při B (probability of A given B) jako
    \[P(A|B) = \frac{P(A\cap B)}{P(B)}.\]
    $Q(A):= P(A|B).$ Pak $(\Omega, \mathcal{F},Q)$ je pravděpodobností prostor.
\end{definition}

\begin{definition}[Zřetězené podmíňování]
    $P(A\cap B) = P(B)P(A|B)$
\end{definition}

\begin{theorem}
    Pokud $A_1,\dots A_n \in \mathcal{F}$ a $P(A_1, \cap \dots \cap A_n)> 0$, tak
    \[P(A_1\cap A_2 \cap \dots \cap A_n) = P(A_1)P(A_2|A_1) ... \text{TODOOT}\]
\end{theorem}
\end{document}
