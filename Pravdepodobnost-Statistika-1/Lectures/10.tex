\documentclass[../main.tex]{subfiles}

%{definition}{Definice}
%{example}{Příklad}
%{intuition}{Intuice}
%{remark}{Poznámka}
%{consequence}{Důsledek}
%{observation}{Pozorování}

\begin{document}
%%%%%%%%%%%%%%%%%%%%%%%%%%%%%%%%%%%%%%%%%%%%%%%%%%%%%%%%%%%%%%%%%%%%%%%%%%%%%%%%
\section{Desátá přednáška}

\subsection{náhodný výběr}
\begin{itemize}
    \item bez vracení \\
    $\Omega = {\text{všechny }n-\text{tice obyvatel ČR}}$\\
    Pro $\omega = (\omega_1,\dots, \omega_n)$ zvolíme $X_i = I(\omega_i$ je levák$)$.
    \item s vracením \\
    $\Omega = \{$všechny $n$-tice obyvatel ČR, mohou se opakovat$\}$\\
    Pro $\omega = (\omega_1,\dots,\omega_n)$ zvolíme $X_i = I(\omega_i$ je levák$)$.
    \item varianty (stratifikovaný výběr)\\
    \textbf{TODO}
\end{itemize}

\subsection{Statistika - model}
\begin{itemize}
    \item nezávislá měření - hodnoty n.n.v. $X_1,\dots,X_n \sim F$ náhodný výběr s distribuční funkcí $F$ s rozsahem $n$
    \item neparametrické modely: povolujeme velkou třídu $F$.
    \item parametrické modely: $F \in \{F_\vartheta: \vartheta \in \Theta\}$
    \item příklady:
    \begin{itemize}
        \item $Pois(\lambda)$ (parametr $\vartheta$)
        \item \textbf{TODO}
    \end{itemize}
\end{itemize}
\subsection{Zkoumané úlohy - cíle konfirmační analýzy}
\begin{enumerate}
    \item bodové odhady
    \item intervalové odhady
    \item testování hypotéz
    \item (linární) regrese
\end{enumerate}
\begin{definition}
    statistika - libovolná funkce náhodného výběru, t. j. \textbf{TODO}
\end{definition}

\begin{example}
    Výběrový průměr a rozptyl\\
    \[\overline{X}_n = \frac{1}{n}\sum^n_{i=1}X_i\]
    \[\overline{S}_n = \frac{1}{n}\sum^n_{i=1}(X_i - \overline{X}_n)^2\]
    \[\widehat{S}_n = \frac{1}{n-1}\sum^n_{i=1}(X_i - \overline{X}_n)^2 \]
\end{example}
\begin{definition}
    Odhad je libovolná statistika.
\end{definition}

\subsection{Vlastnosti bodových odhadů}
\begin{definition}
    Odhad $\widehat{\Theta}_n = \widehat{\Theta}_n(X_1,\dots,X_n)$ parametrů $\vartheta$ je
    \begin{itemize}
        \item nestranný (unbiased), pokud $\vartheta = \mathbb{E}(\widehat{\Theta}_n)$ (pro každé $\vartheta$)
        \item asymptoticky nestranný (asymptotically unbiased) - pokud $\vartheta = \lim_{n \rightarrow \infty} \mathbb{E}(\widehat{\Theta}_n)$
        \item konzistentní (consistent) - pokud $\widehat{\Theta}_n \rightarrow^P \vartheta$
        \item vychýlení (bias) $bias_\vartheta (\widehat{\Theta}_n):=\mathbb{E}(\widehat{\Theta}_n - \vartheta)$
        \item střední kvadratická chyba je $MSE := \mathbb{E}((\widehat{\Theta}_n - \vartheta)^2)$
    \end{itemize}
\end{definition}
\begin{theorem}
    \[MSE = bias_\vartheta(\widehat{\Theta}_n)^2 + var_\vartheta(\widehat{\Theta}_n)\]
    \begin{proof}
        \textbf{TODO}
    \end{proof}
\end{theorem}
\begin{theorem}
    Parametry výběrového momentu a rozptylu\\

    \begin{enumerate}
        \item $\overline{X}_n$ je konzistentní nestranný odhad $\mu = \mathbb{E}X_1 = \mathbb{E}X_2 =\dots$
        \item $\overline{S}_n$ je konzistentní asymptoticky nestranný odhad $\sigma^2$
        \item $\widehat{S}_n$ je konzistentní nestranný odhad $\sigma^2$
    \end{enumerate}
    \begin{proof}
        1. 
        \[\overline{X}_n = \frac{1}{n} X_1 + X_2 + \dots + X_n\]
        $\overline{X}_i$ je nestranný, t. j. $\mathbb{E}\overline{X}_n = \mu$
        \[ = \frac{1}{n} \mathbb{E}X_1 + \mathbb{E}X_2 + \dots + \mathbb{E}X_n = \frac{1}{n} \mu + \mu + \dots + \mu = \mu\]
        $\overline{X}_n$ je konzistentní, t. j. $\overline{X}_n \rightarrow^P \mu$ (slabý zákon velkých čísel) \\
        $var(\overline{X}_n) = frac{\sigma^2}{n} \dots$ Čebyšev\\
        2. 
        \[\overline{S}_n = \frac{1}{n} \sum^n_{i=1}(X_i - \overline{X}_i)^2\]
        \[\mathbb{E}\overline{S}_n = \mathbb{E}\frac{1}{n}\sum^n_{i=1}((X_i - \mu) - (\overline{X}_i - \mu))^2\]
        \[ = \mathbb{E}\frac{1}{n}\sum\left[(X_i - \mu)^2 - 2(X_i - \mu)(\overline{X}_i-\mu) + (\overline{X}_i - \mu)^2\right]\]
        \[ = \mathbb{E}\frac{1}{n}\sum(X_i - \mu) - \mathbb{E} \frac{2}{n} \sum (X_i - \mu)(\overline{X_i-\mu}) + \mathbb{E}(\overline{X}_n - \mu)^2\]
        \[ \frac{1}{n} \sum \mathbb{E}(X_i - \mu)^2 - \mathbb{} (\overline{X}_n - \mu)^2\]
        \[ = \sigma^2 - var(\overline{X}_n) = \sigma^2 - \frac{\sigma^2}{n} = \frac{n-1}{n}\sigma^2\]
        3. 
        \[\widehat{S}_n = \frac{n}{n-1}\overline{S}_n\]
        \[\mathbb{E}\widehat{S}_n = \frac{n}{n-1}\mathbb{E}\widehat{S}_n = \sigma^2 >> \widehat{S}_n \text{ je nestranný odhad.}\]
    \end{proof}
\end{theorem}
Je lepší $\widehat{S}_n$ nebo $\overline{S}_n$?\\
$\rightarrow \widehat{S}_n$ je nestranný, $\overline{S}_n$ ne.\\

\subsection{Metoda momentů}
\begin{itemize}
    \item $m_r(\vartheta) := \mathbb{E}(X^r)$ pro $X \sim F_\vartheta \dots r$-tý momentu
    \item $\widehat{m_r(\vartheta)}:= \frac{1}{n} \sum^n_{i=1}X^r_i$ pro náhodný výběr $X_1,\dots,X_n$ z $F_\vartheta$ \dots $r$-tý výběrový moment
\end{itemize}
\begin{theorem}
    $\widehat{m_r(\vartheta)}$ je nestranný konzistentní odhad pro $m_r(\vartheta)$.
    \begin{proof}
        \[\mathbb{E}\widehat{m_r(\vartheta)} = \frac{1}{n} \sum \mathbb{E}(X^r_i) = \frac{1}{n}\sum \mathbb{E}(X^r_i) = m_r(\vartheta)\]
    \end{proof}
\end{theorem}
\begin{example}
    $X_1,\dots,X_n \sim Bern(p) \dots X_i = $"$i$-tý člověk je levák"\\
    $\vartheta = p \in \left[0,1\right]$\\
    $m_1(\vartheta) = \mathbb{E}X_1 = \vartheta$\\
    $\widehat{m_r(\vartheta)} = \frac{1}{n}(X_1+\dots+X_n) = \overline{X}_n$
\end{example}

\subsection{Metoda maximální věrohodnosti (maximal likelihood, ML)}
\begin{itemize}
    \item náh. výběr $X = (X_1,\dots,X_n)$ z modelu s parametrem $\vartheta$
    \item možný výsledek $x = (x_1,\dots,x_n)$
    \item \dots sdružená pravděpodobnostní funkce $p_X(x;\vartheta)$
    \item \dots sdružená hustota $f_X(x;\vartheta)$
    \item věrohodnost (likelihood) $L(x;\vartheta)$ značí $p_X$ nebo $f_X$
    \item normálně: máme pevné $\vartheta$, a $L(x;\vartheta)$ je funkce $x$
    \item teď: máme pevné $x$ a $L(x;\vartheta)$ je funkce $\vartheta$
\end{itemize}
\begin{itemize}
    \item Metoda MV (ML): volíme takové $\vartheta$, pro které je $L(x;\vartheta)$ maximální
    \item definujeme také $\mathcal{l}(x;\vartheta) = log(L(x;\vartheta))$
    \item díky nezávislosti je
\end{itemize}
\[L(x;\vartheta) = p(x_1;\vartheta)p(x_2;\vartheta)\dots p(x_n;\vartheta)\]
\[\mathcal{l}(x;\vartheta) = \sum^n_{i=1} log p(x_i;\vartheta)\]
\[ O = \mathcal{l}'(x;\vartheta) = \sum \frac{1}{p(x_i;\vartheta) \textbf{TO DO}}\]
\end{document}
