\documentclass[../main.tex]{subfiles}

%{definition}{Definice}
%{example}{Příklad}
%{intuition}{Intuice}
%{remark}{Poznámka}
%{consequence}{Důsledek}
%{observation}{Pozorování}

\begin{document}
%%%%%%%%%%%%%%%%%%%%%%%%%%%%%%%%%%%%%%%%%%%%%%%%%%%%%%%%%%%%%%%%%%%%%%%%%%%%%%%%
\section{Šestá přednáška}

\begin{definition}[Kvantilová funkce]
    Pro náhodnou veličinu $X$ definujeme \textit{kvanitlovou funkci} 
    $Q_X : [0,1] \rightarrow \mathbb{R}$ pomocí
    \[Q_X(p) := min \{x \in \mathbb{R} : p \leq F_X(x)\}\]
    \begin{enumerate}
        \item Pokud $F_X$ je spojitá, tak $Q_X = F^{-1}_X.$
        \item Obecně platí: $Q_X(p) \leq x \Leftrightarrow p \leq F_X(x)$.
        \item $Q_X(\frac{1}{2})$ = medián (pozor, když  $F_X$ nená rostoucí)
        \item Pokud $F_X$ je spojitá, tak $Q_X = F^{-1}_X$.
    \end{enumerate}
\end{definition}

\begin{definition}[Spojitá náhodná veličina]
    N.v. $X$ se nazývá spojitá (\textit{continuous}) pokud existuje nezáporná reálná funkce $f_X$ tak, že
    \[F_X(x) = P(X \leq x) = \int^x_{-\infty} f_X(t)dt\]
    \begin{enumerate}
        \item Alternativně: máme zadanou funkci $f \geq 0$ s $\int^\infty_\infty f = 1$.
        \item Vybereme náhodný bod pod grafem $f$.
        \item Označíme jeho souřadnice $(X,Y)$.
        \item Pak je $X$ n.v. s hustotou $f$.
    \end{enumerate}
\end{definition}

\begin{theorem}[Práce s hustotou]
    Nechť spojitá n.v. $X$ má hustotu $f_X$. Pak
    \begin{enumerate}
        \item $P(X=x) = 0 \forall x \in \mathbb{R}$.
        \item $P(a\leq X \leq b) = \int^b_a f_x(t) dt \forall a,b \in \mathbb{R}$.
        \item V důsledku taky platí (pro rozumnou množinu A):
            \[P(X\in A) = \int_A f_X(t) dt\]
    \end{enumerate} 
\end{theorem}
\begin{proof}
    \[2 \implies 1 : P(x\leq X\leq x) = \int^x_x f = 0\]
    \[2 :  P(a < X \leq b) = P(X\leq b) - P(X\leq a) = F_X(b) - F_X(a) = \int^b_{-\infty} f - \int^a_{-\infty} f\]
    \[P(a\leq X \leq b) = \lim_{n\rightarrow \infty} P(a-\frac{1}{n} < X \leq b) = \lim \int^b_{a-\frac{1}{n}} f = \int^b_a f\]
\end{proof}

\begin{definition}[Střední hodnota spojité n.v.]
    Nechť spojitá n.v. $X$ má hustotu $f_X$. Pak její střední hodnota (\textit{expectation, expected value, mean})
    je označováná $\mathbb{E}(X)$ a definována
    \[\mathbb{E}(X) = \int^\infty_{-\infty} xf_X(x)dx\]
    pokud integrál má smysl, t.j. pokud se nejedná o typ $\infty - \infty$.
    \begin{enumerate}
        \item Analogie s výpočtem těžiště tyče ze znalosti hustoty
        \item Diskretizace.
    \end{enumerate}
\end{definition}

\begin{theorem}[LOTUS]
    Pokud $X$ je spojitá n.v. s hustotou $f_X$ a $g$ reálná funkce, tak
    \[\mathbb{E}(g(X)) = \int^\infty_{-\infty}g(x) f_X(x)dx\]
    pokud integrál má smysl. (Důkaz pomocí substituce v integrálu)
\end{theorem}

\begin{theorem}[Linearita střední hodnoty]
    Pro $X_1,\dots,X_n$ diskrétní nebo spojité n.v. platí
    \[\mathbb{E}(X_1 + \dots + X_n) = \mathbb{E}(X_1) + \dots + \mathbb{E}(X_n)\]
\end{theorem}

\begin{definition}[Rozptyl spojité n.v.]
    \[\mathbb{E}(X) = \int^\infty_{-\infty} xf_X(x)dx\]
    \[\mathbb{E}(X^2) = \int^\infty_{-\infty} x^2f_X(x)dx\]
    Označíme-li $\mu = \mathbb{E}(X),$ tak
    \[var(X) := \mathbb{E}((X-\mu)^2) = \int^\infty_{-\infty} (x-\mu)^2 f_X(x)dx\]
\end{definition}

\begin{theorem}
    Pro spojité n.v. platí $var(X) = \mathbb{E}(X^2) - (\mathbb{E}(X))^2$\\
\end{theorem}
\begin{proof}
    (Důkaz jako pro diskrétní n.v.)
\end{proof}

\begin{theorem}[Rozptyl součtu]
    Pro $X_1,\dots,X_n$ \textit{nezávislé} diskrétní nebo spojité n.v. platí
    \[var(X_1 + \dots + X_n) = var(X_1) + \dots + var(X_n).\]
\end{theorem}
\begin{proof}
    Triviální.
\end{proof}

\begin{definition}[Uniformní rozdělení]
    N.v. $X$ má uniformní rozdělení na intervalu $[a,b]$, píšeme 
    $X \sim U(a,b)$, pokud $f_X(x) = \frac{1}{b-a}$ pro $x \in [a,b] \& f_X(x) = 0$ jinak.
\end{definition}

\begin{definition}[Exponenciální rozdělení]
    \[ F_X(x) =
    \begin{cases}
        0 & \dots x \leq 0\\
        1-e^{-\lambda x} & \dots x \geq 0
    \end{cases}
    \]
\end{definition}
\begin{remark}
    $X$ modeluje např. čas před příchodem dalšího telefonního hovoru do callcentra,
    dotazu na webserver,
    čas do dalšího blesku v bouřce atd.
\end{remark}

\begin{remark}
    Souvislost $X \sim Exp(\lambda)$ a  $Y \sim Geom(p)$
\begin{enumerate}
    \item $P(X > x) = e^{-\lambda x}$ pro $x > 0$
    \item $P(Y > n) = (1-p)^n$ pro $n \in \mathbb{N}$
\end{enumerate}
\end{remark}

\begin{definition}[Standardní normální rozdělení]
    {\color{white} x}

    \begin{enumerate}
        \item $\phi(x) = \frac{1}{\sqrt{2\pi}}e^{\frac{-x^2}{2}}$
        \item $\Phi(x)$ - primitivní funkce k $\phi$
        \item Standardní normální rozdělení $N(0,1)$ má hustotu $\phi$ a distribuční funkci $\Phi$.
        \item Pokud $Z \sim N(0,1)$, tak $\mathbb{E}(Z) = 0$ a $var(Z) = 1$.
    \end{enumerate}
\end{definition}

\begin{definition}[Obecné normální rozdělení]
    {\color{white} x}

    \begin{enumerate}
        \item Pro $\mu,\sigma \in \mathbb{R}, \sigma > 0$ položíme $X = \mu + \sigma \dot Z$, kde $Z \sim N(0,1)$.
        \item Píšeme $X \sim N(\mu, \sigma^2)$ - obecné normální rozdělení
        \item Normální rozdělení $N(\mu,\sigma^2)$ má hustotu $\frac{1}{\sigma}\phi \left( \frac{x-\mu}{\sigma} \right)$
    \end{enumerate}

    \[\Phi(z) = P(Z \leq z) = P(X \leq \mu + \sigma z) = F_X(\mu + \sigma z)\]
\end{definition}

\begin{remark}[Odolnost vůči součtu]
    Pokud $X_1,\dots,X_k$ jsou n.n.v., kde $X_i \sim N(\mu_i, \sigma^2_i)$, pak
    \[X_1 + \dots + X_k \sim N(\mu, \sigma^2),\]
\end{remark}

\begin{remark}[Normální rozdělení - klíčové vlastnosti]
    {\color{white} x}

    \begin{enumerate}
        \item Pravidlo $3\sigma (68-95-99.7$ rule)\\
        $X \sim N(\mu, \sigma^2)$\\
        $P(\mu-\sigma \leq X \leq \mu + \sigma) = 68 \%$\\
        $2\sigma = 95$\\
        $3\sigma = 99.7$
        
        \item Centrální limitní věta
    \end{enumerate}
\end{remark}

\end{document}
