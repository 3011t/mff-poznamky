\documentclass[../main.tex]{subfiles}

%{definition}{Definice}
%{example}{Příklad}
%{intuition}{Intuice}
%{remark}{Poznámka}
%{consequence}{Důsledek}
%{observation}{Pozorování}

\begin{document}
%%%%%%%%%%%%%%%%%%%%%%%%%%%%%%%%%%%%%%%%%%%%%%%%%%%%%%%%%%%%%%%%%%%%%%%%%%%%%%%%
\section{Čtvrtá přednáška}

\begin{theorem}
    Věta\\

    Nechť $X,Y$ jsou diskrétní n.v. a $a,b \in \mathbb{R}$.
    \begin{enumerate}
        \item Pokud $P(X\geq 0) = 1$ a $\mathbb{E}(X) = 0$, tak $P(X=0)=1$.
        \item Pokud $\mathbb{E}(X) \geq 0$ tak $P(X\geq 0) > 0.$
        \item $\mathbb{E}(a X + b) = a \mathbb{E}(X) + b$.
        \item $\mathbb{E}(X+Y) = \mathbb{E}(X) + \mathbb{E}(Y)$
    \end{enumerate}

    \begin{proof}
        1.\\
        \[\mathbb{E}(X) = \sum_{x \in X} x  P(X = x) = 0  P(X = 0) + \sum_{x > \& x \in X} x  P(X = 0) = 
        \sum_{x<0 \& x\in X} x P(X=x) = 0\]
        \[\implies \forall x > 0 : P(x = x) = 0 \implies P(X = 0) = 1\]

        2.\\
        \[\mathbb{E}(X) = \sum x P(X\geq x) = 0\]
        \[\text{kdyby ne: } P(x\geq 0) = 0, \text{všechny členy v sumě by byly záporné...spor}\]

        3.\\
        \[\mathbb{E}(a X+b) = \sum_{x\in X} (a x + b)P(X=x) = a\sum xP(X=x) + b\sum P(X=x)\]

        4.\\
        \[\mathbb{E}(X+Y) = \sum_{\omega}(X(\omega)+Y(\omega))P({\omega}) = \sum X(\omega)P({\omega})
        + \sum Y(\omega)P({\omega}) = \mathbb{E}(X) + \mathbb{E}(Y)\]
    \end{proof}
\end{theorem}

\begin{theorem}
    Nechť $X$ je diskrétní n.v. nabývajíci jen hodnot z $\mathbb{N} = {0,1,2,s}$. Pak platí
    \[\mathbb{E}(X) = \sum^\infty_{n = 0} P(X>n).\]
    \begin{proof}
        \[\mathbb{E}(X) = \sum^\infty_{k=0} kP(X=k) = \sum^\infty_{k=0} \sum^{k-1}_{n=0} P(X=k)\]
        \[= \sum_{n=0}^\infty \sum_{k=n+1}^\infty P(X=k) = \sum \sum P({\omega \in \Omega : X(\omega) = k})\]
        \[= \sum P({\omega \in \Omega : X(\omega) > n}) = \sum^\infty_{n=0} P(X>n)\]
    \end{proof}
\end{theorem}

\begin{definition}
    Rozptyl\\

    Rozptyl (\textit{variance}) n.v. $X$ nazveme číslo $\mathbb{E}((X-\mathbb{E}(X))^2)$.\\
    Značíme jej $var(X)$. \dots (kvadratické měření odchylky)
    \begin{enumerate}
        \item Směrodatná odchylka (standard deviation) $\sigma_X = \sqrt{var(X)}$
        \begin{remark}
            "stejné jednotky jako X"
        \end{remark}
        \item Měří, jak je daleko "typicky" $X$ od $\mathbb{E}(X)$. Mohli bychom to měrit i jinak 
        (např. $\mathbb{E}(|X-\mathbb{E}(X)|)$, ale rozptyl je výhodnější).
    \end{enumerate}
\end{definition}
\begin{theorem}
    Věta\\
    \[var(X) = \mathbb{E}(X^2) - \mathbb{E}(X)^2\]
    \begin{proof}
        \[\mu = \mathbb{E}(X)\]
        \[var(X) = \mathbb{E}((X - \mu)^2) = \mathbb{E}(X^2-2\mu + \mu^2) = \mathbb{E}(X^2) - 2\mu \mathbb{E}(X) + \mu^2\]
        \[ = \]
    \end{proof}
\end{theorem}

\begin{definition}
    Podmíněná střední hodnota\\

    Pokud $X$ je diskrétní n.v. a $P(B) > 0$, tak podmíněná střední hodnota 
    $X$ za předpokladu $B$ (conditional expectation of $X$ given by $B$)
\end{definition}
\begin{theorem}
    Věta o úplné střed. hodnotě\\

    Pokud $B_1,B_2,\dots$ je rozklad $\Omega$ a $X$ je d.n.v., tak
    \[\mathbb{E}(X) = \sum_i \mathbb{E}(X | B_i) P(B_i)\]
    kdykoliv má součet smysl. (Sčítance s $P(B_i) = 0$ považujeme za 0.)
    \begin{proof}
        \[\mathbb{E}(X) = \sum_i P(B_i)\mathbb{E}(X | B_i)\]
        \[= \sum_i P(B_i) \sum_x xP(X=x | B_i)\]
        \[= \sum_x x (\sum_i P(B_i)P())\]
    \end{proof}

    Rozbor všech možností
    $X \sim Geom(p)$\\
    $B_1 = S \dots$ první pokus úspěšný\\
    $B_2 = B^C_1 = F \dots$ první pokus neúspěšný\\
    \[\mathbb{E}(X) = P(S)\mathbb{E}(X|S) + P(F)\mathbb{E}(X|F)\]
    \[= p1 + (1-p)(\mathbb{E}(X+1))\]
    \[p\mathbb{E}(X) = p + (1-p) = 1\]
    \[\mathbb{E}(X) = \frac{1}{p}\]
\end{theorem}
\begin{theorem}
    Parametry rozdělení - Bernoulliho\\

    Pro $X\sim Bern(p)$ je 
    \begin{enumerate}
        \item $\mathbb{E}(X) = p$
        \item $var(X) = p(1-p)$
    \end{enumerate}
    \begin{proof}
        $\mathbb{E}(X) = 0P(X=0)+1P(X=1) = P(X=1) = p$\\
        $var(X) = \mathbb{E}(X-p)^2 = (0-p)^2P(X=0) + (1-p)^2P(X=1) = p(1-p)(p+(1-p)) = p(1-p)$
    \end{proof}
\end{theorem}

\begin{theorem}
    Parametry rozdělení - binomické\\

    Pro $X\sim Bin(n,p)$ je
    \begin{enumerate}
        \item $\mathbb{E}(X) = np$
        \item $var(X) = np(1-p)$
    \end{enumerate}
    \begin{enumerate}
        \item První postup: $X = \sum^n_{i=1}X_i$, kde $X_i = [\text{i-tý hod uspěl}]$
        \[\mathbb{E}(X_i) = P(X_i = 1) = p\]
        \item Druhý postup:
        \[\mathbb{E}(X) = \sum^n_{k=0} kP(X=k) = \sum^n_{k=0}\binom{n}{k}p^k(1-p)^{n-k}\]
        \[\sum_{k=1}^n pn \binom{n-1}{k-1}p^{k-1}(1-p)^{(n-1) - (k-1)}\]
        \[ = pn (p + (1-p))^{n-1} = np\]
    \end{enumerate}
\end{theorem}

\begin{theorem}
    Parametry rozdělení - hypergeometrické\\

    Pro $X\sim Hyper(N,K,n)$ je
    \begin{enumerate}
        \item $\mathbb{E}(X) = n\frac{K}{N}$
        \item $var(X) = n\frac{K}{N}(1-\frac{K}{N})\frac{N-n}{N-1}$
    \end{enumerate}
    \begin{enumerate}
        \item První postup: $X = \sum^n_{i=1}X_i$, kde $X_i = [\text{i-tý míček červený}]$
        \[\mathbb{E}(X_i) = P(X_i = 1) = \frac{K}{N}\]
        \item Druhý postup:
        \[\mathbb{E}(X) = \sum^K_{j=1} Y_j\text{, kde } Y_j = [\text{byl vytažen (z n tahů) míček s číslem j}]\]
        \[\mathbb{E}(Y_j) = P(Y_j = 1) = \frac{n}{N} \]
        \[= \frac{\binom{N-1}{n-1}}{\binom{N}{n}} = \frac{n}{N}\]
    \end{enumerate}
    \textbf{SKONTROLOVAŤ, STRATIL SOM SA :Sadge:}
\end{theorem}

\begin{theorem}
    Parametry rozdělení - geometrické\\

    Pro $X\sim Geom(p)$ je
    \begin{enumerate}
        \item $\mathbb{E}(X) = n\frac{1}{p}$
        \item $var(X) = \frac{1-p}{p^2}$
    \end{enumerate}
\end{theorem}

\begin{theorem}
    Parametry rozdělení - hypergeometrické\\

    Pro $X\sim Hyper(N,K,n)$ je
    \begin{enumerate}
        \item $\mathbb{E}(X) = \lambda$
        \item $var(X) = \lambda$
    \end{enumerate}
    \[P(X=k) = \frac{\lambda^k}{k!}e^{-\lambda}\]
    \[\mathbb{E}(X) = \sum k \frac{\lambda^k}{k!}e^{-\lambda} = 1\]
    \[\lambda \sum \frac{\lambda^{k-1}}{(k-1)!}e^{-\lambda} = \lambda\]    
\end{theorem}

\begin{definition}
    Základní popis náhodných vektorů
    \begin{enumerate}
        \item $X,Y$ - \textbf{nestihol som, lebo preskočil slide za 2 sekundy :) boli tam asi 4 odrážky}
    \end{enumerate}
\end{definition}
\begin{definition}
    Nejaká definícia, nesithol som, lebo som rantoval
\end{definition}
\begin{definition}
    Marginální rozdělení
    \begin{enumerate}
        \item Máme-li dáno $p_{X,Y},$ jak zjistit rozdělení jednotlivých složek, t.j. $p_X$ a $p_Y$ ?
    \end{enumerate}
    \begin{proof}
        Nechť $X,Y$ jsou diskrétní n.v. Pak:
        \[p_X(x) = P(X=x) = \sum_{Y\in Im(Y)} P(X = x \& Y=y) = \sum_{Y\in Im(Y)} p_{X,Y}(x,y)\]
        \[p_Y(y) = P(Y=y) = \sum_{X\in Im(X)} P(X = x \& Y=y) = \sum_{Y\in Im(Y)} p_{X,Y}(x,y)\]
    \end{proof}
\end{definition}
\begin{definition}
    Nezávislost náhodných veličin\\

    Diskrétní n.v. $X,Y$ jsou nezávislé (independent) pokud pro každé $x,y \in \mathbb{R}$ jsou jevy 
    $\{X=x\}$ a $\{Y=y\}$ nezávislé. To nastane právě když
    \[P(X=x,Y=y) = P(X=x)P(Y=y).\]
\end{definition}
\begin{theorem}
    Součin nezávislých n.v.\\

    Pro nezávislé diskrétní n.v. $X,Y$ platí
    \[\mathbb{XY} = \mathbb{E}(X)\mathbb{E}(Y)\]
    \begin{proof}
        \[\mathbb{E}(XY) = \sum_{x\in Im(X), y \in Im(Y)} P(X=x, Y=y)\]
        \[= \sum_x x P(X=x) \sum_y yP(Y=y) = \mathbb{E}(X)\mathbb{E}(Y)\]
    \end{proof}
\end{theorem}
\end{document}
