\documentclass[../main.tex]{subfiles}

%{definition}{Definice}
%{example}{Příklad}
%{intuition}{Intuice}
%{remark}{Poznámka}
%{consequence}{Důsledek}
%{observation}{Pozorování}

\begin{document}
%%%%%%%%%%%%%%%%%%%%%%%%%%%%%%%%%%%%%%%%%%%%%%%%%%%%%%%%%%%%%%%%%%%%%%%%%%%%%%%%
\section{Druhá přednáška}

\subsection{Opakování}

\begin{enumerate}
    \item definice pravděpodobnostního prostoru ($\Omega,\mathcal{F},P)$: dva axiomy,
    \item \textbf{naivní} pravděpodobnostní prostor: $\Omega$ konečná, $\mathcal{F} = \mathcal{P}(\Omega)$\\
    $P(A):=|A|/|\Omega|$
    \item \textbf{diskrétní} pravděpodobnostní prostor: $\Omega = {\omega_1,\omega_2,...}$,\\
    $\mathcal{F} = \mathcal{P}(\Omega),\sum_i p_i = 1$\\
    $P(A) := \sum_{i:\omega_i \in A} p_i$
    \item \textbf{geometrický} pravděpodobnostní prostor:\\
    $\omega \subseteq \mathbb{R}^d$ s konečným objemem,\\
    $P(A) := V_d(A)/V_d(\Omega)$
    \item pravděpodobnostní prostor \textbf{spojitý s hustotou}:\\
    $\Omega \subseteq \mathbb{R}^d$ s funkcí $f$, kde $\int_\Omega f = 1$,\\
    $P(A) := \int_A f$
\end{enumerate}

V pravděpodobnostním prostoru $(\Omega, \mathcal{F},P)$ platí pro $A,B \in \mathcal{F}$
\begin{enumerate}
    \item $P(A^c) = 1 - P(A)$ ... $(A^c = \Omega \backslash A)$
    \item $A \subseteq B \implies P(A) \leq P(B)$
    \item $P(A\cup B) = P(A) + P(B) - P(A \cap B)$ ... PIE
    \item $P(A_1 \cup A_2 \cup \dots) \leq \sum_i P(A_i)$ (subaditivita, Booleova nerovnost)
    \item Definujeme podmíněnou pravděpodobnost (pro $P(B) > 0$).
    \[P(A|B) = \frac{P(A\cap B)}{P(B)}.\]
    \item $Q(A) = P(A|B)$ splňuje axiomy pro pravděpodobnost.
    \[P(\emptyset | B) = 0\]
    \[P(\Omega | B) = \frac{P(B)}{P(B)} = 1\]
    \begin{align*}
        P(A_1 \circ A_2 | B) &= \frac{P((A_1\circ A_2)\cup B)}{P(B)} = \frac{P((A_1\cap B) \cup (A_2 \cap B))}{P(B)}\\ &= \frac{P(A_1\cap B)}{P(B) + \frac{P(A_2 \cap B)}{P(B)}}
        = P(A_1 | B) + P(A_2 | B)
    \end{align*}
\end{enumerate}

\subsection{Podmíněná pravděpodobnost}
\begin{definition}[Zřetězené podmíňování]
    \[P(A\cup B) = P(B)P(A|B)\]
\end{definition}

\begin{theorem}
    Pokud $A_1,\dots,A_n \in \mathcal{F}$ a $P(A_1\cap \dots \cap A_n) > 0$, tak
    \[P(A_1\cap A_2 \cap \dots \cap A_n) = \]
    \[P(A_1)P(A_2|A_1)P(A_3|A_1\cap A_2) \dots P(A_n | \bigcap^{n-1}_{i=1} A_i)\]
\end{theorem}
\begin{proof}
    indukcí % Alright then, keep your secrets.
\end{proof}


\begin{example}
    Vytáhneme 3 karty z balíčku 52 karet. Jaká je P(žádné srdce)?

    $A_i =$ i-tá karta není srdce
    \[P(A_1\cap A_2 \cap A_3) = P(A_1) \times P(A_2 | A_1) P(A_3 | A_2\cap A_1) = \frac{13*3}{52} \times \frac{13*3 -1}{51} \times \frac{13*3 - 2}{50}\]
    \[\frac{\text{\#dobrých}}{\text{\#všech}} = \frac{\binom{39}{3}}{\binom{52}{3}}\]
\end{example}

\begin{definition}
    Spočetný systém množin $B_1,B_2,..\in \mathcal{F}$ je rozklad (partition) $\Omega$, Pokud
    \begin{enumerate}
        \item $B_i \cap B_j = \emptyset$ pro $i \neq j$ a
        \item $\bigcup_i B_i = \Omega$.
    \end{enumerate}
\end{definition}

\begin{theorem}[Věta o úplné pravděpodobnosti]
    = Rozbor všech možností
    \\ Pokud $B_1,B_2,...$ je rozklad $\Omega$ a $A \in \mathcal{F},$ tak
    \[P(A) = \sum_i P(B_i)P(A|B_i)\]
    (sčítance s $P(B_i) = 0$ považujeme za $0$).

    \[A = (A\cap B_1) \cup (A\cap B_2) \cup \dots\]
    (sjednocení disjunktních množin)
    \[P(A) = \sum_i P(A\cap B_i) = \sum_i P(B_i)P(A|B_i)\]
\end{theorem}

\begin{example}
    Máme tři mince: P+O, P+P, O+O. Jaká je pravděpodobnost, že padne orel?

    \noindent
    Označíme $M_1, M_2, M_3$ pro P+O, P+P, O+O.
    
    \[P(O) = P(M_1)P(O|M_1) + P(M_2)P(O|M_2) + P(M_3)P(O|M_3)\]
    \[ = \frac{1}{3}\times \frac{1}{2} + \frac{1}{3}\times 0 + \frac{1}{3}\times 1 = \frac{1}{2}\]

    \noindent
    Rychlejší je vypsat si strom a pak posčítat výsledné jevy
\end{example}

\begin{example}[Gambler's ruin - zbankrotování hazardního hráče.]
    Máme $a$ korun, náš protihráč $b$ korun. Hrajeme opakovaně spravedlivou hru o 1kč,\\
    dokud někdo nepřijde o všechny peníze. Jaká je pravděpodobnost, že vyhrajeme?
\end{example}
\begin{proof}
    \[P_a = P(\text{z této pozice vyhrajeme})\]
    \[P_0 = 0, P_n = 1 \dots (a+b = n)\]
    \[P(\text{výhra}|\text{1. kolo výhra})P(\text{1. kolo výhra})\]
    \[+ P(\text{výhra}|\text{1. kolo prohra})P(\text{1. kolo prohra})\]
    \[\text{výhra}\implies P_{a+1}, \text{prohra}\implies P_{a-1}\]
    \[P_a = \frac{P_{a+1}}{2} + \frac{P_{a-1}}{2}\]
    \[ \Leftrightarrow \]
    \[P_a - P_{a-1} = P_{a+1} - P_a = \Delta\]
    \[1 = P_n = P_0 + n*\Delta \implies \Delta = \frac{1}{n}\]
    \[P_a = \frac{a}{a+b} = \frac{a}{n}\]
\end{proof}

\begin{theorem}[Bayesova Věta]
    Pokud $B_1,B_2,\dots $ je rozklad $\Omega, A \in \mathcal{F}, P(A) > 0$ a $P(B_j) > 0$, tak
    \[P(B_j)| A) = \frac{P(B_j)P(A|B_j)}{P(A)} = \frac{P(B_j)P(A|B_j)}{\sum_i P(B_i)P(A|B_i)}.\]
    (sčítance s $P(B_i) = 0$ považujeme za $0$).
\end{theorem}
\begin{proof}
    \[P(B_j | A)P(A) = P(B_j)P(A|B_j)\]
    \[P(A\cap B_j) = P(B_j \cap A)\]
\end{proof}


\begin{example}
    $N=$ nemocný, $T=$ testovaný, specif. $P(N|T)$, sens. $P(T|N)$.
    \[P(N|T) = \frac{P(N)P(T|N)}{P(N)P(T|N)+ P(N^c)P(T|N^c)} = \frac{p*0.8}{p*0.8 + (1-p)*0.01}\]
    \[p = 0.001 \dots 7\%\]
    \[p = 0.0016 \dots 56\% \dots \text{momentální stav testování}\]
    \[p = 0.05 \dots 80\%\] 
\end{example}

\begin{definition}
    Jevy $A,B \in \mathcal{F}$ jsou nezávislé (independenet) pokud $P(A\cap B) = P(A)P(B)$.\\
    Pak také platí $P(A|B) = P(A)$, pokud $P(B)>0$.
\end{definition}

\begin{definition}
    Jevy $\{A_i : i \in I\}$ jsou (vzájemně) nezávislé, pokud pro každou konečnou množinu $J \subseteq I$
    \[P\left(\bigcap_{i\in J} A_i\right) = \prod_{i\in J} P(A_i).\]
    Pokud podmínka platí jen pro dvouprvkové množiny $J$, nazýváme jevy $\{A_i\}$ po dvou nezávislé (pairwise independent).
\end{definition}

\begin{definition}
    Nechť pro množiny z prostoru jevů platí
    \[A_1 \subseteq A_2 \subseteq A_3 \subseteq \dots\]
    a $A = \bigcup^\infty_{i=1}A_i$. Pak platí
    \[P(A) = lim_{i\rightarrow \infty} P(A_i).\]
    \begin{proof}
        \[A = A_1 \cup (A_2\backslash A_1)\cup (A_3\backslash A_2) \cup \dots\]
        \[P(A) = P(A_1) + P(A_2\backslash A_1) + P(A_3\backslash A_2) + \dots\]
        \[lim_{i\rightarrow \infty} (P(A_1) + \dots + P(A_i \backslash A_{i-1})) = lim_{i\rightarrow \infty P(A_i)}.\]
    \end{proof}
 
    $A_n \subset {P,O}^\mathbb{N}, A_n = $ mezi prvními $n$ hody padl aspoň jednou orel.
    \[P(A) = P(\geq 1 \hspace{1mm} \text{orel v} \hspace{1mm} \infty \hspace{1mm} \text{hodech}) = lim_{n\rightarrow \infty} \dots = 1\]
\end{definition}

\begin{definition}[Náhodná veličina]
    Mějme pravděpodobnostní prostor $(\Omega, \mathcal{F}, P).$ Funkci $X : \Omega \rightarrow \mathbb{R}$
    nazveme diskrétní náhodná veličina, pokud $I_m(X)$ je spočetná množina a pokud pro všechna reálna $x$ platí
    \[\{\omega \in \Omega : X(\omega) = x\} \in \mathcal{F}.\]
\end{definition}

\begin{definition}
    Pravděpodobnostní funkce diskrétní náhodné veličiny $X$ je funkce $p_X : \mathbb{R} \rightarrow [0,1]$ taková, že 
    \[p_X(x) = P(X = x) = P(\{\omega \in \Omega : X(\omega) = x\})\]
\end{definition}

\begin{definition}
    $\sum_{x\in I_m(X)}p_X(x) = 1$
\end{definition}

\begin{definition}
    $S:= I_m(X) 1, Q(A):=\sum_{x\in A} p_X(x)$\\
    $(S,\mathcal{P}(S),Q)$ je diskrétní pravděpodobnostní prostor.
\end{definition}

\begin{definition}
    Pro $S = \{s_i : i \in I\}$ spočetnou množinu reálných čísel a $c_i \in [0,1]$ $\sum_{i\in I} c_i =1$ existuje pravděpodobnostní
    prostor a diskrétní n.v. $X$ na něm taková, že $p_X(s_i) = c_i$ pro $i \in I$.
\end{definition}
\end{document}
