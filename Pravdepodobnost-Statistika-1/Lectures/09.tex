\documentclass[../main.tex]{subfiles}

%{definition}{Definice}
%{example}{Příklad}
%{intuition}{Intuice}
%{remark}{Poznámka}
%{consequence}{Důsledek}
%{observation}{Pozorování}

\begin{document}
%%%%%%%%%%%%%%%%%%%%%%%%%%%%%%%%%%%%%%%%%%%%%%%%%%%%%%%%%%%%%%%%%%%%%%%%%%%%%%%%
\section{Devátá přednáška}

\subsection{Nerovnosti, které známe z minula}
\begin{itemize}
    \item Markovova
    \[X\geq 0 \implies P(X\geq a\mathbb{E}(X))\leq \frac{1}{a}\]
    \item Čebyševova
    \[P(|X-\mathbb{E}(X))|\geq a\sigma_X)\leq \frac{1}{a^2}\]
    \item Chernoffova ($\sigma_X = \sqrt{n}$)
    \[X = \sum^n_{i=1}X_i,X_i = \pm 1 \implies P(|X-\mathbb{E}(X)|> a\sigma_X)\leq 2e^{-a^2/2}\]
\end{itemize}

\subsection{Slabý zákon velkých čísel}
\begin{theorem}
    Nechť $X_1,\dots ,X_n$ jsou stejné rozdělené n.n.v. se střední hodnotou $\mu$ a rozptylem $\sigma^2$.
    Označme $S_n = (X_1 + \dots + X_n)/n$.
    Pak pro každé $\varepsilon > 0$ platí
    \[\lim_{n\rightarrow \infty} P(|S_n - \mu | \geq \varepsilon) = 0.\]
    Říkáme, že posloupnost $S_n$ konverguje k $\mu$ v pravděpodobnosti, píšeme $S_n \rightarrow^P \mu$.
\end{theorem}
\begin{proof}
    \begin{center}
        \[\mathbb{E}S_n = \mathbb{E}\frac{X_1 + \dots + X_n}{n} = \frac{\mathbb{E}X_1 + \dots + \mathbb{E}X_n}{n}
         = \frac{\mu + \dots + \mu}{n} = \mu\]\\
        \[var(S_n) = var \left(\frac{X_1 + \dots + X_n}{n}\right) = \frac{var(X_1) + \dots + var(X_n)}{n^2} 
        = \frac{\sigma^2 + \dots + \sigma^2}{n^2} = \frac{\sigma^2}{n}\]\\
        \[P(|S_n - \mathbb{E}S_n | \geq a\sigma_{S_n}) \leq \frac{1}{a^2} = \frac{1}{\left(\frac{\varepsilon \sqrt{n}}{\sigma}\right)} =
        \frac{\sigma^2}{\varepsilon^2 n} \rightarrow_{n\rightarrow \infty} 0\]\\
    \end{center}
\end{proof}

\subsection{Centrální limitní věta}
\begin{theorem}[Centrální limitní věta]
    Nechť $X_1,\dots,X_n$ jsou stejně rozdělené n.n.v se střední hodnotou $\mu$ a rozptylem $\sigma^2$. Označme
    \[Y_n = ((X_1 + \dots + X_n) - n\mu)/(\frac{n}\sigma).\]
    Pak $Y_n \rightarrow^d N(0,1)$. Neboli, pokud $F_n$ je distribuční funkce $Y_n$, tak
    \[\lim_{n \rightarrow \infty}F_n(x) = \Phi(x) \forall x\in \mathbb{R}.\]
    Říkáme, že posloupnost $Y_n$ konverguje k $N(0,1)$ v distribuci.
    \\ \textbf{Doplnit tri grafy z prezentace}
\end{theorem}

\subsection{Momentová vytvořující funkce}
\begin{definition}[Momentová vytvořující funkce]
    Pro náhodnou veličinu $X$ označíme
    \[M_X(t) = \mathbb{E}(e^{t_{X}}).\]
    Funkci $M_X(t)\dots$
    \textbf{DOPLNIT}
\end{definition}

\subsection{Statistika}
\begin{example}[1. Počet leváků]

    \begin{itemize}
        \item \#L = 6 = 14\%
        \item \#P = 37 = 87\%
        \item spolu: 43 = 100\%
    \end{itemize}
    Tipujeme, že je 4 - 12\% leváků v ČR.

    \begin{remark}

        otázky statistiky $\rightarrow$ co můžeme z výsledků v malém vzorku odvodit o výsledcích v celé skupině
        \begin{itemize}
        \item bodové odhady \dots 14\%
        \item intervalové odhady \dots (10\%, 20\%)
        \end{itemize}
        Obtíže statistiky $\rightarrow$ otázky typu
        \begin{itemize}
            \item máme reprezentativní vzorek?
            \item je otázka dobře formulovaná?
        \end{itemize}
    \end{remark}
\end{example}

\begin{example}[2. Doba běhu programu]
    {\color{white} x}
    \begin{itemize}
        \item $X_1,\dots,X_n \sim F$ n.n.v., $F$ je jejich distribuční funkce
    \end{itemize}
    \begin{definition}
        Empirická distribuční funkce (empirical CDF) je definována
        \[\hat{F}_n(x) = \frac{\sum^n_{i=1}I(X_i \leq x)}{n},\]
        kde $I(X_i \leq x) = 1$ pokud $X_i \leq x$ a 0 jinak.
    \end{definition}
\end{example}

\begin{theorem}
    Pro pevné $x$ platí
    \begin{itemize}
        \item $\mathbb{E}(\hat{F}_n(x)) = F(x)$
        \item $var(\hat{F}_n(X)) = \frac{F(x)(1-F(x))}{n}$
        \item $\hat{F}_n(x)$ konverguje k $F(x)$ v pravděpodobnosti, píšeme $\hat{F}_n(x)\rightarrow^P F(x)$.
    \end{itemize}
    \begin{proof}
        Slabý zákon velkých čísel:
        \[\mathbb{E}\hat{F}_n(x) = \mathbb{E}S_n = \mathbb{E}I(X_i \leq x) = P(X_i \leq x) = F(x)\]
        \[var(\hat{F}_n(x)) = \frac{var(X'_1)}{n}\]
        \[X'_i \sim Bern(p) \dots p=F(x)\]
    \end{proof}
\end{theorem}

\subsection{Empirická distribuční funkce - Dvoretzky-Keifer-Wolfowitz (DKW)}

\begin{theorem}[Empirická distribuční funkce]
    Nechť $X_1,\dots,X_n \sim F$ jsou n.n.v., $\hat{F}_n$ jejich empirická distribuční funkce.
    Nechť $\mathbb{E}(X_i)$ je konečná. Zvolme $\alpha \in (0,1)$ (pravděpodobnost chyby) 
    a označme $\varepsilon = \sqrt{\frac{1}{2n}log \frac{2}{\alpha}}$. Pak platí:
    \[P(\hat{F}_n(x) - \varepsilon) \leq F(x) \leq \textbf{TO DO}\]
\end{theorem}

\subsection{Intro - explorační analýza dat (exploratory data analysis)}
\begin{itemize}
    \item posbíráme data ( a dáme pozor na systémové chyby - nezávislost, nezaujatost\dots)
    \item různé tabulky (třeba v Excelu a spol.)
    \item vhodné obrázky: histogram, krabicový diagram (boxplot) atd.
\end{itemize}

\end{document}
