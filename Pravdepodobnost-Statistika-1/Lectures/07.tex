\documentclass[../main.tex]{subfiles}

%{definition}{Definice}
%{example}{Příklad}
%{intuition}{Intuice}
%{remark}{Poznámka}
%{consequence}{Důsledek}
%{observation}{Pozorování}

\begin{document}
%%%%%%%%%%%%%%%%%%%%%%%%%%%%%%%%%%%%%%%%%%%%%%%%%%%%%%%%%%%%%%%%%%%%%%%%%%%%%%%%
\section{Sedmá přednáška}


\begin{definition}[Cauchyho rozdělení]

    hustota $f(x) = \frac{1}{\pi (1+x^2)}$
    nemá střední hodnotu!

    \begin{remark}
        \[\int^\infty_{-\infty} f = \frac{1}{\pi} \int^\infty_{-\infty} \frac{1}{1+x^2} = \frac{1}{\pi}\left[arctg(x)\right]^\infty_{-\infty} = 1\]
        \[\mathbb{E}X = \int^\infty_{-\infty} x f(x) = \int^\infty_0 \frac{2x}{2\pi (1+x^2)} + \int^0_{-\infty}\frac{x}{\pi (1+x^2)}\]
        \[\left[\frac{1}{2\pi}log(1+x^2)\right]^\infty_0 + \left[\frac{1}{x\pi} log(1+x^2)\right]\]
        \[\infty - 0 + 0- \infty = \infty - \infty ?!\]
    \end{remark}
    
\end{definition}

\begin{definition} [Gamma rozdělení]

    $Gamma(w,\lambda)$, gamma rozdělení s parametry $w > 0$ a $\lambda > 0$ má hustotu
    \[f(x) = 0 \text{ pro } x \leq 0 \hspace{7mm} \& \hspace{7mm} \frac{1}{\Gamma(w)}\lambda^w x^{w-1} e^{-\lambda x} \text{ pro } x\geq 0\]
    kde $\Gamma(w) = (w-1)! = \int^\infty_0 x^{w-1}e^{-x}dx$

    Pro $w = 1$ dostáváme znovu exponenciální rozdělení ... $\frac{1}{0!}\lambda^1 e^{-\lambda x}$\\
    Pokud $X_1,\dots ,X_n$ jsou n.n.v s rozdělením $Exp(\lambda)$, tak $X_1 + \dots + X_n \sim Gamma(n, \lambda)$.  
\end{definition}
\begin{theorem}
    Nechť $X$ je n.v. s distribční funkcí $F_X = F$, nechť $F$ je spojitá a rostoucí. Pak $F(X) \sim U(0,1)$.\\
\end{theorem}
\begin{proof}
    \[F_Y(y) = P(F(X) \leq y) = 0 \text{ pro } y < 0 \& 1 \text{ pro } y \geq 1\]
    \[\text{pro } y \in (0,1) P(X \leq x) \implies \text{ stejné jevy } \dots = F(x) = y\]
\end{proof}

\begin{theorem}
    Nechť $F$ je funkce \textit{"typu distribuční funkce"} : neklesajíci zprava spojitá funkce s $lim_{x\rightarrow -\infty} F(x) = 0$ a $\lim_{x \rightarrow \infty} F(x) = 1$.

    Nechť $Q$ je odpovídající kvantilová funkce. Nechť $U \sim U(0,1), X = Q(U)$. Pak $X$ má distribuční funkci $F$.
\end{theorem}
\begin{proof}
    \[F_X(x) = P(Q(U) \leq x)\]
    \begin{remark}
        \[Q(p) = inf\{x: F(X) \geq p\} \implies Q(p) \leq x \Leftrightarrow F(x) \geq p\]
    \end{remark}
    \[F_X(x) = P(U \leq F(x)) = F(x)\]
\end{proof}

\begin{example}
    \[F(x) = 1-e^{-\lambda x} \dots Exp(\lambda)\]
    \[Q(p) = \frac{log(1-p)}{-\lambda} > 0\]
    \[U \sim U(0,1) \dots \frac{log(1-U)}{-\lambda} \sim Exp(\lambda)\]
\end{example}

\begin{definition}
    Sdružená distribuční funkce (Joint cdf)\\

    Pro n.v. $X,Y$ na pravděpodobnostním prostoru $(\Omega,\mathcal{F},P)$ definujeme
    jejich sdruženou distribuční funkci (joint cdf) $F_{X,Y} : \mathbb{R}^2 \rightarrow [0,1]$ předpisem
    \[F_{X,Y}(x,y) = P(\{\omega \in \Omega : X(\omega) \leq x \& Y(\omega) \leq y\}).\]
    \begin{enumerate}
        \item Formální podmínka: potřebujeme $\{X\leq x \& Y \leq Y\} \in \mathcal{F}$, jinak $(X,Y)$ není náhodný vektor.
        \item Mohli bychom definovat i pro více než dvě n.v. \dots $F_{X_1,\dots ,X_n(x_1,\dots ,x_n)} = P(X_1 \leq x_1 \& \dots X_n \leq x_n)$.
        \item Můžeme odsud odvodit pravděpodobnost obdélníku:
        \[P(X \in (a,b] \& Y \in (c,d]) = F(b,d) - F(b,c) - F(a,d) + F(a,c)\]
    \end{enumerate}
\end{definition}

\begin{definition}
    Sdružená hustota (Joint pdf)\\

    Často můžeme sdruženou distribuční funkci psát jako integrál pomocí nezáporné funkce $f_{X,Y}$
    \[F_{X,Y}(x,y) = \int^x_{-\infty}\int^y_{-\infty} f_{X,Y}(s,t)dsdt.\]
    Pak nazýváme n.v. $X,Y$ sdruženě spojité. Funkce $f_{X,Y}$ je jejich sdružená hustota.

    Jako u jednorozměrného případu může být $f_{X,Y} > 1$.

    Stejně jako u jednorozměrného případu můžeme pak pomocí hustoty vyjádřit i další pravděpodobnosti,
    pro \textit{"rozumnou množinu A"}.
    \[P((X,Y)\in A) = \int_A f_{X,Y}(x,y)dxdy\]
    \[\int_{\mathbb{R}^2} f_{X,Y} = 1\]
\end{definition}

\begin{remark}
    \[f_{X,Y}(x,y) = \frac{\partial^2}{\partial x \partial y} F_{X,Y}(x,y)\]
    \[f_{X,Y}(x) \doteq \frac{P(x\leq X \leq x + \Delta_x \& y\leq Y \leq y + \Delta_y)}{\Delta_x \Delta_y}\]
    \[P((X,Y)\in A) = \int_A f = \int^{x+\Delta_x}_x \int^{y+\Delta_y}_y f_{X,Y}(s,t)dsdt = f_{x,s}(x,s)\Delta_x \Delta_y\]
\end{remark}

\begin{definition}
    LOTUS\\

    Analogicky jako v diskrétním případu platí pro střední hodnotu funkce dvou n.v.
    \[\mathbb{E}(g(X,Y)) = \int^\infty_{-\infty} \int^\infty_{-\infty} g(x,y)f_{X,Y}(x,y)dxdy.\]
    A tak jako v diskrétním případu odsud odvodíme
    \[\mathbb{E}(aX+bY+c) = a\mathbb{E}(X) + b\mathbb{E}(Y) + c\]
    \[\mathbb{E}(g(X,Y)) = \int \int g(x,y)f_{X,Y}(x,y) = \int \int ax f(x,y) + \int \int by f(x,y) + c \int^\infty_{-\infty}f(x,y)=\]
    \[a \int x \int f_{X,Y}(x,y)dydx + b \int y \int f(x,y)dydx + 1 =\]
    \[a \int xf_X(x) + b \int yf_Y(y) + 1 = \]
    \[ a\mathbb{E}(X) + b\mathbb{E}(Y) + 1\]
\end{definition}

\begin{definition}
    Nezávislost spojitých náhodných veličin\\

    Libovolné náhodné veličiny nazveme nezávislé (independent), pokud jevy $\{X\leq x\}$ a $\{Y\leq y\}$ jsou nezávislé pro libovolná $x,y \in \mathbb{R}$. Ekvivalentně:
    \[P(X\leq x, Y\leq y) = P(X\leq x)P(Y\leq y),\]
    \[F_{X,Y}(x,y) = F_X(x)F_Y(y)\]
\end{definition}
\begin{theorem}
    Nechť $X,Y$ mají sdruženou hustotu $f_{X,Y}$. Následující tvrzení jsou ekvivalentní:
    \begin{enumerate}
        \item $X,Y$ jsou nezávislé
        \item $f_{X,Y}(x,y) = f_X(x) f_Y(y)$
    \end{enumerate}
\end{theorem}
\begin{proof}
    \[\implies : f_{X,Y}(x,y) = \frac{\partial^2}{\partial x \partial y} F_{X,Y}(x,y) = \]
    \[F'_X F'_y = f_X(x)f_Y(y)\]
    \textbf{doplniť druhú implikáciu (nestihol som, zo slidov)}
\end{proof}

Vícerozměrné normální rozdělení\\

\begin{enumerate}
    \item $\varphi(t) = \frac{e^{-t^2/2}}{\sqrt(2\pi)}$
    \item $f(t_1,\dots, t_n) = \varphi(t_1)\varphi(t_2)\dots \varphi(t_n) = \frac{e^{-\frac{t^2_1 + \dots + t^2_n}{2}}}{\sqrt{2\pi}}$
    \item $f(t_1,\dots, t_n = (2\pi)^{-\frac{n}{2}}e^{-\frac{r^2}{2})}$, kde $r^2 = t^2_1 + \dots + t^2_n$ je radiálně symetrická funkce.
    \item Nechť $Z = (Z_1,\dots, Z_n)$ má hustotu $f$.
    \item $Z_1,\dots,Z_n$ jsou n.n.v, $Z_i \sim N(0,1)$
    \item $Z/||Z||$ je uniformně náhodný bod na $n$-rozměrné sfěře
    \item skalární součin $Z$ s libovolným jednotkovým vektorem je $N(0,1)$
    \item $\left< u, Z \right> = \sum^n_{i=1} u_i Z_i$ má také rozdělení $N(0,1)$
\end{enumerate}

Vícerozměrné normální rozdělení obecné\\

\begin{enumerate}
    \item Obecněji můžeme vzít náhodný vektor s hustotou $ce^{Q(t)}$, kde $c > 0$ je vhodná konstanta a $Q(t)$ je obecná kvadratická funkce.
    \item Používá se ve strojovém učení.
    \item Souřadnice nejsou nezávislé.
\end{enumerate}
\end{document}
