\documentclass[../main.tex]{subfiles}

%{definition}{Definice}
%{example}{Příklad}
%{intuition}{Intuice}
%{remark}{Poznámka}
%{consequence}{Důsledek}
%{observation}{Pozorování}

\begin{document}
%%%%%%%%%%%%%%%%%%%%%%%%%%%%%%%%%%%%%%%%%%%%%%%%%%%%%%%%%%%%%%%%%%%%%%%%%%%%%%%%
\section{Jedenáctá přednáška}

\subsection{Intervalové odhady}
\begin{itemize}
    \item místo jednoho čísla s nejistým významem vypočítáme z dat interval $\left[\widehat{\Theta}^-, \widehat{\Theta}^+\right]$
\end{itemize}

\begin{definition}[Konfidenční interval]
    Nechť $\widehat{\Theta}^-, \widehat{\Theta}^+$ jsou n.v.,
    které závisí na náhodném výběru 
    $X = (X_1,\dots,X_n)$ z distribuce $F_{\vartheta}$. Tyto n.v. určují intervalový odhad, též konfidenční interval o spolehlivosti
    $1-\alpha$ (\textit{confidence interval}) pokud 
    \[P(\widehat{\Theta}^- \leq \vartheta \leq \widehat{\Theta}^+) \geq 1 - \alpha\]
    \begin{itemize}
        \item tohle jsou tzv. oboustranné odhady
        \item jednostranný odhad: $[\widehat{\Theta}^-, \infty)$ nebo $(-\infty, \widehat{\Theta}^+]$
    \end{itemize}
\end{definition}

\begin{theorem}
    $X_1,\dots,X_n$ je náhodný výběr z $N(\vartheta, \sigma^2)$.\\
    $\sigma$ známe, $\vartheta$ chceme určit, $\alpha \in (0,1)$.\\
    Nechť $\Phi(z_{\alpha/2}) = 1 - \alpha / 2$. Zvolíme $\widehat{\Theta}_n := \widehat{X}_n$.
    \[C_n := \left[\widehat{\Theta}_n - z_{\alpha/2}\frac{\sigma}{\sqrt{n}}, \widehat{\Theta}_n + z_{\alpha/2}\frac{\sigma}{\sqrt{n}}\right]\]
    Pak $P(C_n \ni \vartheta) = 1 - \alpha$.
\end{theorem}
\begin{proof}
    \[C_n \ni \vartheta \Leftrightarrow |\widehat{\Theta}_n - \vartheta | \leq z_{\alpha/2}\frac{\sigma}{\sqrt{n}}\]
    \[\Leftrightarrow \left|\frac{\widehat{\Theta}_n - \vartheta}{\sigma/\sqrt{n}}\right| \leq z_{\alpha/2}\]
    \[ \frac{\widehat{\Theta}_n - \vartheta}{\sigma/\sqrt{n}} = Z \sim N(0,1)\]
    \[P(C_n \ni \vartheta) = P(|Z| \leq z_{\alpha/2}) = \Phi(z_{\alpha/2}) - \Phi(-z_{\alpha/2})\]
    \[= (1- \alpha/2) - (+ \alpha/2) = 1 - \alpha\]
\end{proof}

\begin{theorem}
    $X_1,\dots,X_n$ je náhodný výběr z rozdělení se střední hodnotou $\vartheta$, rozptylem $\sigma^2$.\\
    $\sigma$ známe, $\vartheta$ chceme určit, $\alpha \in (0,1)$.\\
    Nechť $\Phi(z_{\alpha/2}) = 1 - \alpha / 2$. Zvolíme $\widehat{\Theta}_n := \widehat{X}_n$.
    \[C_n := \left[\widehat{\Theta}_n - z_{\alpha/2}\frac{\sigma}{\sqrt{n}}, \widehat{\Theta}_n + z_{\alpha/2}\frac{\sigma}{\sqrt{n}}\right]\]
    Pak $\lim_{n \rightarrow \infty} P(C_n \ni \vartheta) = 1 - \alpha$.
\end{theorem}
\begin{proof}
    Centrální limitní věta.
\end{proof}

\begin{definition}[Studentovo rozdělení]
    {\color{white} x}
    \begin{itemize}
        \item $\overline{X}_n = \frac{1}{n} \sum^n_{i=1} X_i \dots$ výběrový průměr
        \item $\widehat{S}^2_n = \frac{1}{n-1}\sum^n_{i=1} (X_i - \overline{X}_n)^2$ \dots výběrový rozptyl
    \end{itemize}
    \begin{itemize}
        \item Nechť $X_1, \dots X_n \sim N(\mu,\sigma^2)$
        \item Pak $\frac{\overline{X}_n - \mu}{\widehat{S}^2_n/\sqrt{n}} \sim N(0,1)$
        \item Studentovo $t$-rozdělení s $n-1$ stupni volnosti je rozdělení n.v.
            $\frac{\overline{X}_n - \mu}{\widehat{S}^2_n/\sqrt{n}}$
        \item Distribuční funkci budeme značit $\Psi_{n-1}$. Je v tabulkách, v $R:pt(x,n-1)$\textbf{TODO}
    \end{itemize}
\end{definition}

\begin{theorem}
    $X_1,\dots,X_n$ je náhodný výběr z $N(\vartheta,\sigma^2)$.\\
    $\vartheta$ chceme určit, $\sigma$ neznáme, $\alpha \in (0,1).$ Nechť
    \[\Psi_{n-1}(z_{\alpha/2}) = 1 - \alpha/2,\ \widehat{\Theta}_n =
    \widehat{X}_n,\ \widehat{S}^2_n = \frac{1}{n-1} \sum^n_{i=1} (X_i - \overline{X}_n)^2\]
    \[C_n := \left[\widehat{\Theta}_n - z_{\alpha/2} \frac{\widehat{S}_n}{\sqrt{n}}, \widehat{\Theta}_n + z_{\alpha/2}\frac{\widehat{S}_n}{\sqrt{n}}\right]\]
    Pak $P(C_n \ni \vartheta) = 1 - \alpha$
\end{theorem}

\begin{proof}
    \[P(C_n \ni \vartheta) = P (|Z| \leq z_{\alpha/2}) = \Psi_{n-1}(z_{\alpha/2}) - \Psi_{n-1}(-z_{\alpha/2}) = 1 - \alpha/2 - \alpha/2 = 1-\alpha\]
    \[* Z = \frac{}{} - \text{ st. } t-\text{rozdělení s } n-1.\]
\end{proof}

\subsection{Testování hypotéz}
\begin{itemize}
    \item Je naše mince spravedlivá?
    \item Je naše kostka spravedlivá?
    \item Má vylepšený program kratší dobu běhu než původní?
    \item Je léčba nemoci metodou X dobrá? (Lepší než placebo, lepší než metoda Y,\dots)
    \item Jsou leváci lepší boxeři?
\end{itemize}
\begin{itemize}
    \item dvě hypotézy: $H_0,H_1$
    \item $H_0$ - nulová hypotéza - značí defaultní, konzervativní model (léčba, mince je spravedlivá)
    \item $H_1$ - alternativní hypotéza - značí alternativní model "pozoruhodnost"
\end{itemize}
\begin{example}[Testování hypotéz]
    {\color{white} x}
    \begin{itemize}
        \item Chceme testovat, zda je mince spravedlivá.
        \item Hodíme $n$-krát mincí, orel padne $S$-krát.
        \item Pokud je $|S-n/2|$ moc velké, tak mince není spravedlivá.
    \end{itemize}
\end{example}
\end{document}
