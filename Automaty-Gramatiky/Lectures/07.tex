\documentclass[../main.tex]{subfiles}

%{definition}{Definice}
%{example}{Příklad}
%{intuition}{Intuice}
%{remark}{Poznámka}
%{consequence}{Důsledek}
%{observation}{Pozorování}

\begin{document}
%%%%%%%%%%%%%%%%%%%%%%%%%%%%%%%%%%%%%%%%%%%%%%%%%%%%%%%%%%%%%%%%%%%%%%%%%%%%%%%%
\section{Sedmá přednáška}

    Zásobníkový automat\\

    Zásobníkové automaty jsou rozšířením $\lambda -NFA$ nedeterministických konečných
    automatů s $\lambda$ přechody.

    Přidanou věcí je zásobník. Ze zásobníku můžeme číst, přidávat na vrch a odebírat
    z vrchu zásobníku znak $\in \Gamma$.

    Může si pamatovat neomezené množství informace.

    Zásobníkové automaty definují bezkontextové jazyky.

    Deterministické zásobníkové automaty přijímají jen vlastní podmnožinu bezkontextových jazyků.

    \begin{definition}
        Zásobníkový automat\\

        (PDA) je $P = (Q, \Sigma, \Gamma, \delta, q_0, Z_0, F)$, kde 
        \begin{enumerate}
            \item $Q$ konečná množina stavů,
            \item $\Sigma$ neprázdná konečná množina vstupních symbolů,
            \item $\Gamma$ neprázdná konečná zásobníková abeceda,
            \item $\delta$ přechodová funkce : $Q \times (\Sigma \cup {\lambda})
            \times \Gamma \rightarrow P(_{FIN}(Q\times \Gamma^*))$, $\delta (p,a,X) \ni (q,\gamma)$
            kde $q$ je nový stav a $\gamma$ je řetězec zásobníkových symbolů, který nahradí $X$ na vrcholu zásobníku,
            \item $q_0 \in Q$ počáteční stav
            \item $Z_0 \in \Gamma$ počáteční zásobníkový symbol. Víc na začátku zásobníku není.
            \item $F$ množina přijímajícich (koncových) stavů; může být nedefinovaná.
        \end{enumerate}

        V jednom časovém kroku zásobníkový automat:
        \begin{enumerate}
            \item Přečte na vstupu žádný nebo jeden symbol ($\lambda$ přechody pro prázdný vstup.)
            \item Přejde do nového stavu.
            \item Nahradí symbol na vrchu zásobníku libovolným řetězcem ($\lambda$ odpovídá samotnému \textit{pop},
            jinak následuje \textit{push} jednoho nebo více symbolů).
        \end{enumerate}
    \end{definition}

    \begin{example}
        Zásobníkový automat pro jazyk: $L_{wwr} = {ww^R : w \in (0+1)^*}$.\\

        PDA přijímající $L_{wwr}$:
        \begin{enumerate}
            \item Start $q_0$ reprezentuje odhad, že ještě nejsme uprostřed.
            \item V každém kroku nedeterministicky hádáme:
            \begin{enumerate}
                \item Zůstat $q_0$ (ještě nejsme uprostřed)
                \item Přejít $\lambda$ přechodem do $q_1$ (už jsme viděli střed).
            \end{enumerate}
            \item V $q_0$ přečte vstupní symbol a dá (push) ho na zásobník
            \item V $q_1$ srovná vstupní symbol s vrcholem zásobníku. Pokud se shodují,
            přečte vstupní symbol a umaže (pop) vrchol zásobníku
            \item Když vyprázdníme zásobník, přijmeme vstup, který jsme doteď přečetli.
        \end{enumerate}
    \end{example}

    \begin{example}
        PDA pro $L_{wwr}$\\

        \textbf{TO DO, (slide 143, example 7.2)}
    \end{example}

    \begin{definition}
        Přechodový diagram pro zásobníkový automat\\

        Přechodový diagram pro zásobníkový automat obsahuje:
        \begin{enumerate}
            \item Uzly, které odpovídají stavům $PDA$.
            \item Šipka 'odnikud' ukazuje počáteční stav, dvojité kruhy označují přijímající stavy.
            \item Hrana odpovídá přechodu PDA. Hrana označená $a, X \rightarrow \alpha$ ze stavu
            $p$ do $q$ znamená $\delta(p,a,X) \ni (q,\alpha)$
            \item Konvence je, že počáteční symbol zásobníku značíme $Z_0$.
        \end{enumerate}
    \end{definition}

    \begin{definition}
        Notace zásobníkových automatů\\

        \begin{tabular}{l l}
            $a,b,c,*,+,1,(,)$ & symboly vstupní abecedy\\
            $p,q,r$ & stavy \\
            $u,v,w,x,y,z$ & řetězce vstupní abecedy\\
            $X,Y,E,I,S$ & zásobníkové symboly\\
            $\alpha, \beta, \gamma$ & řetězce zásobníkových symbolů\\
        \end{tabular}
    \end{definition}

    \begin{definition}
        Situace zásobníkového automatu\\

        Situaci zásobníkového automatu reprezentujeme trojicí $(q,w,\gamma)$, kde
        \begin{enumerate}
            \item $q$ je stav,
            \item $w$ je zbývající vstup a 
            \item $\gamma$ je obsah zásobníku (vrch zásobníku je vlevo).
        \end{enumerate}
        Situaci značíme zkratkou $(ID)$ z anglického $instantaneous description$.
    \end{definition}

    \begin{definition}
        $\vdash, \vdash^*$ posloupnosti situací\\
        
        Mějme PDA $P = (Q,\Sigma, \Gamma, \delta, q_0, Z_0, F)$. Definujeme $\vdash_P$ nebo $\vdash$ následovně:
        \begin{enumerate}
            \item Nechť $\delta(p,a,X) \ni (q,\alpha), p, q \in Q, a \in (\Sigma \cup {\lambda}), X \in \Gamma, \alpha \in \Gamma^*$.
            \[\forall w \in \Sigma^*, \beta \in \Gamma^* : (p,aw, X\beta) \vdash (q,w, \alpha \beta).\]
            \item Symboly $\vdash^*_P$ a $\vdash^*$ používáme na označení nuly a více kroků zásobníkového automatu, t. j.
            \begin{enumerate}
                \item $I \vdash^* I$ pro každou situaci I
                \item $I \vdash^* J$ pokud existuje situace $K : I \vdash K \& K \vdash^* J$.
            \end{enumerate}
            \item Čteme $I \vdash^* J$ situace I vede na situaci J, $I \vdash J$ situace I bezprostředně vede na situaci J.
        \end{enumerate}
    \end{definition}

    \begin{definition}
        Jazyk přijímaný koncovým stavem, prázdným zásobníkem\\

        Mějme zásobníkový automat $P = (Q,\Sigma, \Gamma, \delta, q_0, Z_0, F)$. Pak $L(P)$ je:
        \[L(P) = \{w : (q_0, w, Z_0) \vdash^*_P (q, \lambda, \alpha) \text{ pro nějaké $q\in F$ a libovolný řetězec } \alpha \in \Gamma^*; w \in \Sigma^*\}\]

        jazyk akceptovaný prázdným zásobníkem $N(P)$ definujeme:
        \[L(P) = \{w : (q_0, w, Z_0) \vdash^*_P (q, \lambda, \lambda) \text{ pro nějaké $q\in Q$ }; w \in \Sigma^*\}\]
    \end{definition}

    \begin{example}
        Zásobníkový automat z předchozího příkladu akceptuje $L_{wwr}$ koncovým stavem.
    \end{example}

    \begin{example}
        $P' \equiv P$ z předchozího příkladu, jen změníme instrukci, aby umazala poslední symbol\\

        $\delta(q_1, \lambda, Z_0) = {(q_2,Z_0)}$ nahradíme\\
        $\delta(q_1, \lambda, Z_0) = {(q_2,\lambda)}$\\
        nyní $L(P') = N(P') = L_{wwr}$.        
    \end{example}

    \begin{example}
        If-else přijímané prázdným zásobníkem\\

        \textbf{TO DO, 7.5, 149 slide}
    \end{example}

    \begin{example}
        Přijímaní koncovým stavem\\

        \textbf{TO DO, 7.6, 149 slide}
    \end{example}

    \begin{theorem}
        Nečtený vstup a dno zásobníku $P$ neovlivní výpočet\\

        Mějme $PDA P = (Q,\Sigma, \Gamma, \delta, q_0, Z_0, F)$ a $(p,x,\alpha)\vdash^*_P (q,y,\beta)$.
        Potom pro libovolné slovo $w \in \Sigma^*$ a $\gamma \in \Gamma^*$ platí : $(q,xw,\alpha \gamma)\vdash^*_P (q,yw, \beta \gamma)$.
        Speciálně pro $\gamma = \lambda$ a/nebo $w = \lambda$.

        \begin{proof}
            Indukcí podle počtu situací mezi $(p, xw, \alpha \gamma)$ a $(q, yw, \beta \gamma)$. Každý krok
            $(p,x,\alpha) \vdash^*_P (q,y,\beta)$ je určen bez $w$ a/nebo $\gamma$. Proto je možný i se symboly na konci vstupu / dně zásobníku.
        \end{proof}
    \end{theorem}

    \begin{remark}
        \[\text{Pro } PDA P = (Q,\Sigma,\Gamma, \delta, q_0, Z_0, F), (p, xw, \alpha) \vdash^*_P (q, yw, \beta) : (p,x,\alpha)\vdash^*_P (q,y,\beta).\]

        Pro zásobník ale obdoba neplatí. PDA  může zásobníkové symboly $\gamma$ použít a zase je tam naskládat (push).
        $L = {0^i1^i0^j1^j}$, situace $(p, 0^{i-j}1^i0^j1^j, 0^jZ_0)\vdash^* (q,1^j, 0^j Z_0)$, mezitím vyčistíme zásobník k $Z_0$
    \end{remark}

    \begin{theorem}
        Od přijímajíciho stavu k prázdneému zásobníku\\

        Mějme $L = L(P_F)$ pro nějaký PDA.\\

        $P_F = (Q,\Sigma, \Gamma, \delta_F, q_0, Z_0, F)$. Pak existuje $PDA P_N$ takový, že $L = L(P_N)$.

        \begin{proof}
            Nechť $P_N = (Q \cup {p_0, p}, \Sigma, \Gamma \cup {X_0}, \delta_N, p_0, X_0)$, kde:
            \begin{enumerate}
                \item $\delta_N(p_0,\lambda, X_0) = {(q,Z_0X_0)}$ start,
                \item $\forall(q\in Q, a \in \Sigma \cup {\lambda}, Y \in \Gamma) : \delta_N(q,a,Y) = \delta_F(q,a,Y)$ simulujeme
                \item $\forall(q \in F, Y \in \Gamma \cup {X_0}), \delta_N(q,\lambda, Y) \ni {(p,\lambda)}$ přijmout pokud $P_F$ přijíma,
                \item $\forall (Y \in \Gamma \cup {X_0}), \delta_N(p,\lambda, Y) = {(p,\lambda)}$ vyprázdníme zásobník. 
            \end{enumerate}
            Pak $w\in N(P_N) \Leftrightarrow w\in L(P_F).$
        \end{proof}
    \end{theorem}

    \begin{theorem}
        Od prázdného zásobníku ke koncovému stavu\\

        Pokud $L = N(P_N)$ pro nějaký PDA $P_N = (Q,\Sigma, \Gamma, \delta_N, q_0, Z_0),$ pak existuje PDA $P_F : L = L(P_F)$.
        
        \begin{proof}
            \[P_F = (Q\cup \{p_0,p_f\},\Sigma, \Gamma \cup \{X_0\}, \delta_F, p_0, X_0, {p_f})\]

            kde $\delta_F$ je:
            \begin{enumerate}
                \item $\delta_F(p_0, \lambda, X_0) = {(q_0,Z_0 X_0)}$ start
                \item $\forall (q\in Q, a \in \Sigma \cup {\lambda}, Y \in \Gamma), \delta_F(q,a,Y) = \delta_N(q,a,Y)$.
                \item Navíc, $\delta_F(q,\lambda, X_0) \ni (p_f, \lambda) \forall q\in Q$. 
            \end{enumerate}
            Chceme ukázat, že $w \in L(P_N) \Leftrightarrow w \in L(P_F)$.
            \begin{enumerate}
                \item (if) $P_F$ přijíma následovně:
                $(p_0, w, X_0) \vdash_{P_F}(q_0,w,Z_0 X_0)\vdash^*_{P_F=N_F}
                (q,\lambda, X_0) \vdash_{P_F}(p_f,\lambda,\lambda).$
                \item (only if) Do $p_f$ nelze dojít jinak než předchozím bodem.
            \end{enumerate}
        \end{proof}
    \end{theorem}

    \begin{theorem}
        $L(CFG), L(PDA), N(PDA)$\\

        Následující tvrzení jsou ekvivalentní:
        \begin{enumerate}
            \item Jazyk $L$ je bezkontextový, t.j. generovaný $CFG$,
            \item Jazyk $L$ je přijímaný nějakým zásobníkovým automatem koncovým stavem.
            \item Jazyk $L$ je přijímaný nějakým zásobníkovým automatem prázdným zásobníkem.
        \end{enumerate}

        \begin{proof}
            \textbf{TO DO, doplniť graf z 153 (theorem 7.1)}    
        \end{proof}
    \end{theorem}

    \begin{theorem}
        Algoritmus: Konstrukce $PDA$ z $CFG G$\\

        Mějme $CFG$ gramatiku $G = (V,T,P,S)$. Konstruujeme $PDA P = ({q}, T,V\cup T, \delta, q,S)$.
        \begin{enumerate}
            \item Pro neterminály $A \in V, \delta(q,\lambda, A) = {(q,\beta) : A \rightarrow \beta \text{ je pravidlo }G}$.
            \item Pro každý terminál $a \in T, \delta(q,a,a) = {(q,\lambda)}$.
        \end{enumerate}
    \end{theorem}
    \begin{example}
        Konverze gramatiky\\

        \textbf{TODO, strana 154, example 7.7}
    \end{example}

    \begin{theorem}
        Přijímání prázdným zásobníkem ze CFG\\

        Pro PDA P konstruovaný z CFG G algoritmem výše je N(P) = L(G).\\
        \begin{proof}
            \begin{enumerate}
                \item Levá derivace: $E \Rightarrow E * E \Rightarrow I * E \Rightarrow a*E \Rightarrow a*I \Rightarrow a*b$
                \item Posloupnost situací: $(q,a*b,E)\vdash (q,a*b,E*E)\vdash (q,a*b,I*E)\vdash (q,a*b,a*E)\vdash (q,*b,*E)\vdash(q,b,E)\vdash(q,b,I)\vdash(q,b,b)\vdash(q,\lambda,\lambda)$.
            \end{enumerate}    

            Pozorování:
            \begin{enumerate}
                \item Kroky derivace simuluje $PDA \lambda$ přepisy zásobníku
                \item odmazávaný vstup u $PDA$ v derivaci zůstává až do konce
                \item až $PDA$ vymaže terminály, pokračuje v přepisech.
            \end{enumerate}


            $w\in N(P) \Leftarrow w\in L(G)$\\

            Nechť $w\in L(G)$, $w$ má levou derivaci $S = \gamma_1, \Rightarrow_{Im} \gamma_2 \Rightarrow \dots \Rightarrow \gamma_n = w.$

            Indukcí podle $i$ dokážeme $(q,w,S)\vdash^*_P (q,v_i,\alpha_i) : \gamma_i = u_i \alpha_i$ je levá sentenciální forma a $u_iv_i = w$.
        \end{proof}
    \end{theorem}
\end{document}