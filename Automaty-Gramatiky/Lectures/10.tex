\documentclass[../main.tex]{subfiles}

%{definition}{Definice}
%{example}{Příklad}
%{intuition}{Intuice}
%{remark}{Poznámka}
%{consequence}{Důsledek}
%{observation}{Pozorování}

\begin{document}
%%%%%%%%%%%%%%%%%%%%%%%%%%%%%%%%%%%%%%%%%%%%%%%%%%%%%%%%%%%%%%%%%%%%%%%%%%%%%%%%
\section{Desátá přednáška}
\subsection{Uzávěrová vlastnosti}
\begin{theorem}
    CFL uzavřené na sjednocení, konkatenaci, uzávěr, reverzi\\

    CFL jsou uzavřené na sjednocení, konkatenaci, iteraci (*), pozitivní iteraci (+), homomorfismus, zrcadlový obraz $w^R$.
    \begin{proof}
        \begin{itemize}
            \item Sjednocení:
            \begin{itemize}
                \item pokud $V_1 \cap V_2 \neq \emptyset$ přejmenujeme neterminály,
                \item přidáme nový symbol $S_{new}$ a pravidlo $S_{new} \rightarrow S_1|S_2$.
            \end{itemize}
            \item Zřetězení $L_1.L_2$
            \begin{itemize}
                \item $S_{new} \rightarrow S_1S_2$ (pro $V_1 \cap V_2 = \emptyset$, jinak přejmenujeme)  
            \end{itemize}
            \item iterace $L* =\bigcup_{i\geq 0} L^i$
            \begin{itemize}
                \item $S_{new} \rightarrow SS_{new}|\lambda$
            \end{itemize}
            \item pozitivní iterace $L^+ = \bigcup_{igeq 1} L^i$
            \begin{itemize}
                \item $S_{new} \rightarrow SS_{new}|S$
            \end{itemize}
            \item zrcadlový obraz $L^R = \{w^R|w\in L\}$
            \begin{itemize}
                \item $X \rightarrow \omega^R$ obrátíme pravou stranu pravidel. 
            \end{itemize}
        \end{itemize}
    \end{proof}
\end{theorem}

\begin{example}
    \begin{itemize}
        \item Jazyk $L = \{0^n1^n2^n|n\geq 1\} = \{0^n1^n2^i | n, i \geq 1\} \cap \{0^i1^n2^n|n,i\geq 1\}$
        není CFL, i když oba členy průniku jsou bezkontextové, dokonce deterministické bezkontextové.
        \[\{0^n1^n2^i|n,i \geq 1\} \rightarrow \{S \rightarrow AC, A \rightarrow 0A1 | 01, C \rightarrow 2C|2\}\]
        \[\{0^i1^n2^n|n,i \geq 1\} \rightarrow \{S \rightarrow AB, A \rightarrow 0A | 0, B \rightarrow 1B2|12\}\]
        \item průnik není CFL z pumping lemmatu. 
    \end{itemize}
    paralelní běh dvou zásobníkových automatů
    \begin{itemize}
        \item řídící jednotky umíme spojit (viz konečné automaty)
        \item čtení umíme spojit (jeden automat může čekat)
        \item bohužel dva zásobníky nelze obecně spojit do jednoho
    \end{itemize}
    dva neomezené zásobníky $\rightarrow$ Turingův stroj, rekurzivně spočetné jazyky $\mathcal{L}_0$ 
\end{example}
\begin{theorem}
    CFL i DCFL jsou uzavřené na průnik s regulárním jazykem\\
    \begin{itemize}
        \item Mějme $L$ bezkontextový jazyk a $R$ regulární jazyk.\\
            Pak $L\cap R$ je bezkontextový jazyk.
        \item Mějme $L$ deterministický CFL a $R$ regulární jazyk.\\
            Pak $L\cap R$ je deterministický CFL.
    \end{itemize}
    \begin{proof}
        \begin{itemize}
            \item zásobníkový a konečný automat můžeme spojit
            \begin{itemize}
                \item FA $A_1 = (Q_1,\Sigma,\delta_1,q_1,F_1)$
                \item PDA přijímaný stavem $M_1 = (Q_2,\Sigma,\Gamma,\delta_2,q_2,Z_0,F_2)$
            \end{itemize}
            \item nový automat $M = (Q_1 \times Q_2, \Sigma, \Gamma, \delta, (q_1,q_2), Z_0, F_1 \times F_2)$
            \begin{itemize}
                \item $((r,s),\alpha)\in \delta((p,q),a,Z)$ právě když 
                \item $ a\neq \lambda: r = \delta_1(p,a)\&(s,\alpha)\in \delta_2(q,a,Z)$ \dots automaty čtou vstup, PDA mění zásobník, FA stojí.
                \item $a = \lambda: (s,\alpha)\in \delta_2(q,\lambda,Z)$
                \item $r = p$
            \end{itemize}
            \item zřejmě $L(M) = L(A_1) \cap L(M_2)$
            \begin{itemize}
                \item paralelní běh automatů.
            \end{itemize}
        \end{itemize}
    \end{proof}
\end{theorem}
\begin{example}
    Substituce\\
    Mějme gramatiku $G = ({E},{a,+,(,)},{E \rightarrow E + E|(E)|a},E)$. Mějme substituci:
    \begin{itemize}
        \item $\sigma(a) = L(G_a), G_a = ({I},{a,b,0,1},{I\rightarrow I0|I1|Ia|Ib|a|b},I),$
        \item $\sigma(+) = {-,\times,:,div,mod}$,
        \item $\sigma(() = {(}$,
        \item $\sigma()) = {)}$.
    \end{itemize}
    \begin{itemize}
        \item $(a+a) + a \in L(G)$
        \item $(a001 - bba)\ast b_1 \in \sigma((a+a)+a)\subset \sigma(L(G))$,
        \item v $\sigma(a)$ chybí $+$ pro ukázku, že $(a+a) + a \notin \sigma(L(G))$.
    \end{itemize}
    Co se stane, když změníme definici:
    \begin{itemize}
        \item $\sigma(() = {(,[}$,
        \item $\sigma()) = {),]}$?
    \end{itemize}
\end{example}
\begin{example}
    Homomorfismus\\
    Mějme gramatiku $G = ({E},{a,+,(,)},{E \rightarrow E+E|(E)|a},E)$. Mějme homomorfismus:
    \begin{itemize}
        \item $h(a) = \lambda$
        \item $h(+) = \lambda$
        \item $h(() = left$
        \item $h()) = right$
    \end{itemize}
    \begin{itemize}
        \item $h((a+a)+a) = leftright,$
        \item $h^{-1}(leftright) \ni (a + +)a.$
    \end{itemize}
\end{example}
\begin{theorem}
    CFL jsou uzavřené na substituci\\

    Mějme CFL jazyk $L$ nad $\Sigma$ a substituci $\sigma$ na $\Sigma$ takovou, že $\sigma(a)$ je CFL $\forall a\in \Sigma$. Pak je 
    i $\sigma(L)$ CFL (bezkontextový).
    \begin{proof}
        \begin{itemize}
            \item Idea: listy v derivačním stromu generují další stromy,
            \item Přejmenujeme neterminály na jednoznačné všude v $G = (V,\Sigma, P,S), G_a = (V_a,T_a,P_a,S_a), a \in \Sigma$.
            \item Vytvoříme novou gramatiku $G = (V',T',P',S)$ pro $\sigma(L)$:
            \begin{itemize}
                \item $V' = V \cup \bigcup_{a\in \Sigma}V_a$,
                \item $T'$
            \end{itemize}
        \end{itemize}
    \end{proof}
\end{theorem}
\begin{example}
    Substituce\\

    
    % \begin{tabular}{}
    %     $L = {a^ib^j|0\leq i\leq j)}$ && $S \rightarrow aSb|Sb|\lambda$\\
    %     $\sigma(a) = L_1 = {c^id^i|i \geq 0}$ && $S_1 \rightarrow cS_1d|\lambda$\\
    %     $TO$ && $DO$\\
    % \end{tabular}
\end{example}
\begin{theorem}
    CFL jsou uzavřené na inverzní homomorfismus\\

    Mějme CFL jazyk $L$ a homomorfismus $h$. Pak $h^{-1}(L)$ je bezkontextový jazyk.
    Je-li $TO DO$
    \begin{proof}
        \begin{itemize}
            \item pro $L$ máme PDA $M = (Q,\Sigma,\delta,q_0,Z_0,F)$ (koncovým stavem)
            \item $h : T \rightarrow \Sigma^*$
            \item definujeme PDA $M' = (Q',T,\Gamma,\delta', [q_0,\lambda],Z_0,F\times {\lambda})$, kde
            \[\textbf{DOPLNIT, slide 196}\]
            Pro deterministický PDA $M$ je i $M'$ deterministický.
        \end{itemize}
    \end{proof}
\end{theorem}
\[197\]
\begin{proof}
    Důkaz sporem\\
    \begin{itemize}
        \item Nechť $L$ je bezkontextový jazyk
        \item $L_1 = \{01^j2^k3^l|j,k,l \geq 0\}$ je regulární Jazyk
        \begin{itemize}
            \item ${S \rightarrow 0B,B \rightarrow 1B|C,C\rightarrow 2C|D,D\rightarrow 3D|\lambda}$
        \end{itemize}
        \item $L\cap L_1 = {01^i2^i3^i|i\geq 0}$ není bezkontextový $\implies$ spor
    \end{itemize}
\end{proof}

\begin{theorem}
    Rozdíl s regulárním jazykem\\

    Mějme bezkontextový jazyk $L$ a regulární jazyk $R$. Pak:
    \begin{itemize}
        \item $L-R$ je CFL.
    \end{itemize}
    \begin{proof}
        $L - R = L \cap \overline{}{R}, \overline{R}$ je regulární.
    \end{proof}
\end{theorem}
\begin{theorem}
    CFL nejsou uzavřené na doplněk ani rozdíl\\
    
    Mějme bezkontextové jazyky $L,L_1,L_2$, regulární jazyk $R$. Pak:
    \begin{itemize}
        \item $\overline{L}$ nemusí být CFL. \dots $L_1 \cap L_2 = \overline{\overline{L_1} \textbf{TODO}}$
    \end{itemize}
\end{theorem}
\subsection{Uzávěrové vlastnosti deterministických CFL}
\begin{itemize}
    \item Rozumné programovací jazyky jsou deterministické CFL.
    \item Deterministické bezkontextové jazyky
    \begin{itemize}
        \item nejsou uzavřené na průnik
        \item jsou uzavřené na průnik s regulárním jazykem
        \item jsou uzavřené na inverzní homomorfismus
    \end{itemize}
\end{itemize}
\begin{theorem}
    Doplněk deterministického CFL je opět deterministický CFL.
    \begin{proof}
        \begin{itemize}
            \item idea: prohodíme koncové a nekoncové stavy
            \item nedefinované kroky ošetříme 'podložkou' na zásobníku
            \item cyklus odhalíme pomocí čítače
            \item až po přečtení slova prochádzí koncové a nekoncové stavy (stačí si pamatovat, zda prošel koncovým stavem.)
        \end{itemize}
    \end{proof}
\end{theorem}
\textbf{Uzávěrové vlastnosti v kostce - doplnit tabulku (202/255-274)}
\begin{remark}
    Ne-uzavřenost deterministických CFL
    \begin{proof}
        Vzhledem k uzavřenosti DCFL na doplněk by byl DCFL i $\overline{L}\cap a*b*c* = {a^ib^jc^k|i=j=k}$,
        o kterém víme, že není CFL (pumping lemma)
    \end{proof}
\end{remark}
\textbf{DCFL nejsou uzavřené na homomorfismus (201/255-274 )}
\end{document}