\documentclass[../main.tex]{subfiles}

%{definition}{Definice}
%{example}{Příklad}
%{intuition}{Intuice}
%{remark}{Poznámka}
%{consequence}{Důsledek}
%{observation}{Pozorování}

\begin{document}
%%%%%%%%%%%%%%%%%%%%%%%%%%%%%%%%%%%%%%%%%%%%%%%%%%%%%%%%%%%%%%%%%%%%%%%%%%%%%%%%
\section{Třetí přednáška}

\begin{definition}[Algoritmus na testování ekvivalnece regulárních jazyků]
    Ekvivalenci regulárních jazyků L,M testujeme následovně:
    \begin{enumerate}
        \item Najdeme $DFA A_L, A_M$ rozpoznávajíci $L(A_L) = L, L(A_M) = M$, $Q_L \cap Q_M = \emptyset$.
        \item Vytvoříme $DFA$ sjednocením stavů a přechodů $(Q_L \cup Q_M, \Sigma, \delta_L \cap \delta_M, q_L, F_L \cap F_M)$;
         zvolíme jeden z počátečních stavů.
        \item Jazyky jsou ekvivalentní právě když počáteční stavy původních DFA jsou ekvivalentní. 
    \end{enumerate}
\end{definition}[Nedeterministické konečné automaty (NFA)]
    Nedeterministický automat může být ve více stavech paralelně. Má schopnost 'uhodnout' něco o vstupu.

    \textbf{ pridať obrázok zo slidov (61. slide)}

\begin{definition}[NFA]
    Nedeterministický konečný automat (NFA) $A = (Q, \Sigma, \delta, S_0, F)$ sestává z:
    \begin{enumerate}
        \item konečné množiny stavů, zpravidla značíme $Q$,
        \item konečné množiny vstupních symbolů, značíme $\Sigma$
        \item přechodové funkce, zobrazení $\delta : Q \times \Sigma \rightarrow \mathcal{P}(Q)$ vracejíci podmnožinu Q.
        \item množiny počátečních stavů $S_0 \subseteq Q$,
        \item množiny koncových (přijímajícich) stavů $F \subseteq Q$.
    \end{enumerate}
\end{definition}

\begin{definition}[Rozšířená přechodová funkce]
    Pro přechodovou funkci $\delta$ NFA je rozšířená přechodová funkce $\delta^*$

    $\delta^* : Q \times \Sigma^* \rightarrow \mathcal{P}(Q)$ definovaná indukcí:

    start: $\delta^*(q,\lambda) = {q}.$
    ind. indukční krok:
    \[\delta^*(q,wx) = \bigcup_{p\in \delta^*(q,w)}\delta(p,x)\]
    t. j. množina stavů, do kterých se mohu dostat posloupností 'správně označených'
\end{definition}

\begin{definition}[Jazyk přijímaný nedeterministickým konečným automatem]
    Mějme NFA $A = (Q,\Sigma, \delta, S_0, F)$, Pak
    \[L(A) = \{w : (\exists q_0 \in S_0) \delta^*(q_0,w \cap F \neq \emptyset)\}\]
    je jazyk přijímaný automatem A.
    
    Tedy $L(A)$ je množina slov $w \in \Sigma^*$ takových, že $\delta^*(q_0,w)$ obsahuje alespoň jeden přijímajíci stav.
\end{definition}

\begin{algorithm}[Podmnožinová konstrukce]
    Podmnožinová konstrukce začíná s NFA $N = (Q_N,\Sigma, \delta_N, S_0, F_N)$. Cílem je popis deterministického
    DFA $D = (Q_D,\Sigma, \delta_D, S_0, F_D)$, pro který $L(N) = L(D)$.
    \begin{enumerate}
        \item $Q_D$ je množina podmnožin $Q_N,Q_D = \mathcal{P}(Q_N)$ (potenční množina).
        \begin{remark}
            Nedosažitelné stavy můžeme vynechat
        \end{remark}
        \item Počáteční stav DFA je stav označený $S_0$, t.j. prvek $Q_D$.
        \item $F_D = \{S : S\in \mathcal{P}(Q_N) \& S\cap F_N \neq \emptyset\}$, tedy $S$ obsahuje alespoň jeden přijímajíci stav $N$.
        \item Pro každé $S \subseteq Q_N$ a každý vstupní symbol $a \in \Sigma$,
        \[\delta_D(S,a) = \bigcup_{p\in S} \delta_N(p,a).\]
    \end{enumerate}
\end{algorithm}

\begin{theorem}[Převod NFA na DFA]
    Pro DFA $D = (Q_D,\Sigma, \delta_D, S_0, F_D)$ vytvořený podmnožinovou konstrukcí z NFA $N = (Q_N,\Sigma, \delta_N,q_0,F_N)$ platí $L(N) = L(D)$.
\end{theorem}
\begin{proof}
    Indukcí dokážeme, že $\delta^*_D(S_0,w) = \delta^*_N(q_0,w).$
\end{proof}

Můžeme přidat ještě tzv. $\lambda$-přechod.

\begin{definition}[$\lambda$-přechod]
    Dovolíme přechody na $\lambda$, prázdné slovo, t.j. bez přečtení vstupního symbolu.
    
    \textbf{doplniť obrázok, 68. slide}
\end{definition}

\begin{definition}[$\lambda$-uzávěr]
    Pro $q\in Q$ definujeme $\lambda$-uzávěr $\lambda CLOSE(q)$ rekurzivně:
    \begin{enumerate}
        \item Stav $q$ je $\lambda CLOSE(q)$.
        \item Je-li $p \in \lambda CLOSE(q)$ a $r\in \delta(p,\lambda)$, pak i $r \in \lambda CLOSE(q)$.
    \end{enumerate}

    Pro $S \subseteq Q$ definujeme $\lambda CLOSE(S) = \bigcup_{q\in S} \lambda CLOSE(q)$. 
\end{definition}

\begin{definition}[Rozšířená přechodová funkce a jazyk přijímaný $\lambda$-NFA]
    Nechť $E = (Q,\Sigma,\delta,S_0,F)$ je $\lambda$-NFA. Rozšířenou přechodovou funkci $\delta^*$ definujeme následovně:
    \begin{enumerate}
        \item $\delta^*(q,\lambda) = \lambda CLOSE(q)$.
        \item indukční krok: $v = wa$, kde $w\in \Sigma^*, a\in \Sigma$.
        \[\delta^*(q,wa) = \lambda CLOSE \left(\bigcup_{p\in \delta^*(q,w)}\delta (p,a)\right)\]
    \end{enumerate}
\end{definition}

\begin{theorem}[Eliminace $\lambda$-přechodů]
    Jazyk $L$ je rozpoznatelný $\lambda$-NFA právě když je $L$ regulární.
\end{theorem}
\begin{proof}
    Pro libovolný $\lambda$NFA $E = (Q_E,\Sigma, \delta_E, S_0, F_E)$ zkonstruujeme DFA $D = (Q_D,\Sigma, \delta_D, q_D, F_D)$
    přijímajíci stejný jazyk jako $E$.
    \begin{enumerate}
        \item $Q_D \subseteq \mathcal{P}(Q_E), \forall S \subseteq Q_E : \lambda CLOSE(S) \in Q_D$. V $Q_D$ může být i $\emptyset$.
        \item $q_D = \lambda CLOSE(S_0)$.
        \item $F_D = \{S : S\in Q_D \& S \cap F_E \neq \emptyset\}$.
        \item Pro $S \in Q_D, a\in \Sigma$ definujeme $\delta_D(S,a) = \lambda CLOSE(\bigcup_{p\in S} \delta(p,a))$. 
    \end{enumerate}
\end{proof}

\begin{definition}[Množinové operace nad jazyky]
    Mějme dva jazyky $L,M$. Definujeme následujíci operace:
    \begin{enumerate}
        \item binární sjednocení $L\cup M = \{w: w\in L \lor w \in M\}$
        \begin{remark}
            Příklad: jazyk obsahuje slova začínajíci $a^i$ nebo tvaru $b^j c^j$.
        \end{remark}
        \item průnik $L\cap M = \{w: w\in L \& w\in M\}$
        \begin{remark}
            Příklad: jazyk obsahuje slova sudé délky končíci na 'baa'.    
        \end{remark}
        \item rozdíl $L - M = \{w: w\in L \& w \notin M\}$
        \item doplněk (komplement) $\overline{L} = -L = \{w: w\in L\} = \sigma^* -L$
        \begin{remark}
            Příklad: jazyk obsahuje slova nekončíci na 'a'.
        \end{remark}
    \end{enumerate}
\end{definition}

\begin{theorem}[de Morganova pravidla]
    \begin{enumerate}
        \item $L\cap M = \overline{\overline{L}\cup \overline{M}}$
        \item $L\cup M = \overline{\overline{L}\cap \overline{M}}$
        \item $L - M = L \cap \overline{M}$
    \end{enumerate}
\end{theorem}

\begin{theorem}[Uzavřenost na množinové operace]
    Mějme regulární jazyky $L,M$. Pak jsou následujíci jazyky také regulární:
    \begin{enumerate}
        \item sjednocení $L\cup M$
        \item průnik $L \cap M$
        \item rozdíl $L - M$
        \item doplněk $\overline{L} = \Sigma^* - L$.
    \end{enumerate} 
\end{theorem}
\begin{proof}
    \begin{enumerate}
        \item Pokud $\delta$ není pro některé dvojice $q,a$ definovaná, přidáme nový nepřijímajíci stav $q_n$ a do něj
        přechod pro vše dříve nedefinované plus $\forall a \in \Sigma \cup {\lambda}: \delta(q_n,x) = q_n$.
        \item Pak stačí prohodit koncové a nekoncové stavy přijímajíciho deterministického FA $F = Q_A - F_A$.
        \item pro rozdíl doplníme funkci $\delta$ na totální.
        \item Zkonstruujeme součinový automat,
        \[Q = (Q_1 \times Q_2, \Sigma, \delta((p_1,p_2),x) = (\delta_1(p_1,x),\delta_2,x)), (q_{0_1},q_{0_2},F) \]
        \item průnik: $F = F_1 \times F_2$
        \item sjednocení: $F = (F_1 \times Q_2) \cup (Q_1 \times F_2)$
        \item rozdíl: $F = F_1 \times (Q_2 - F_2)$. 
    \end{enumerate}
\end{proof}


\begin{definition}[Řetězcové operace nad jazyky]
    {\color{white} x}

    \begin{enumerate}
        \item zřetězení jazyků ... $L.M = \{uv : u \in L \& v \in M\}$, $L.x = L.{x}$ a $x.L = {x}.L$ pro $x \in \Sigma$
        \item mocniny jazyka ... $L^0 = {\lambda}$, $L^{i+1} = L^i.L$
        \item pozitivní iterace ... $L^+ = L^1 \cup L^2 \dots \bigcup_{i\geq 1} L^i$
        \item obecná iterace ... $L^* = L^0 \cup L^1 \cup \dots = \bigcup_{i\geq 0}L^i$, tedy $L^* = L^+ \cup {\lambda}$
        \item otočení jazyka ... $L^R = \{u^R : u\in L\}$
        \item levý kvocient L podle M ... $M\\L = \{u : uv \in L \& u\in M\}$
        \item levá derivace L podle w ... $\partial_wL = \{w\}\\ L$
        \item pravý kvocient L podle M ... $L/M = \{u: uv \in L \& v \in M\}$
        \item pravá derivace L podle w ... $\partial^R_wL=L/\{w\}$.
    \end{enumerate}
\end{definition}

\begin{theorem}
    Jsou-li L, M regulární jazyky, je regulární i $L.M, L^*,L^+, L^R, M\\L a L/M$.
\end{theorem}
\begin{theorem}
    Jsou-li L,M regulární jazyky, je regulární i $L.M$.
\end{theorem}
\begin{proof}
    Vezmeme DFA $A_1 = (Q_1,\Sigma, \delta_1, q_1, F_1)$,
    pak $A_2 = (Q_2,\Sigma, \delta_2,q_2,F_2)$ tak, že $L = L(A_1)$ a $M = L(A_2)$.
    \\ Definujeme Nedeterministický automat $B = (Q\cup {q_0},\Sigma, \delta, q_0,F_2)$ kde:

    \begin{tabular}{l r}
        $Q = Q_1 \cup Q_2$ & předpokládáme různá jména stavů, jinak přejmenujeme, končíme až po přečtení slova z $L_2$ \\        
    \end{tabular}
    \begin{tabular}{c|c|c|c}
        $\delta(q_0,\lambda)$ & $={q_1,q_2}$ & pro $q_1 \in F_1$ & $t.j. \lambda \in L(A_1)$ \\
        $\delta(q_0,\lambda)$ & $={q_1}$ & pro $q_1 \notin F_1$ & $t.j. \lambda \notin L(A_1)$ \\
        $\delta(q_0,x)$ & $=\emptyset$ & pro $x \in \Sigma$ & \space \\
        $\delta(q,x)$ & $={\delta_1(q,x)}$ & pro $q \in Q_1 \& \delta_1(q,x) \notin F_1$ & počítáme v $A_1$\\
        \space & $={\delta_1(q,x),q_2}$ & pro $q \in Q_1 \& \delta_1(q,x) \in F_1$ & nedet. přechod z $A_1$\\
        \space & $={\delta_2(q,x)}$ & pro $q \in Q_2$  & počítáme v $A_2$\\
    \end{tabular}

    Pak $L(B) = L(A_1).L(A_2)$.

    I have zero idea how this should be formatted properly, pls fix later. - 3O11
\end{proof}
\end{document}