\documentclass[../main.tex]{subfiles}

%{definition}{Definice}
%{example}{Příklad}
%{intuition}{Intuice}
%{remark}{Poznámka}
%{consequence}{Důsledek}
%{observation}{Pozorování}

\begin{document}
%%%%%%%%%%%%%%%%%%%%%%%%%%%%%%%%%%%%%%%%%%%%%%%%%%%%%%%%%%%%%%%%%%%%%%%%%%%%%%%%
\section{Jedenáctá přednáška}
\begin{definition}
    Dyckův jazyk\\

    Dyckův jazyk $D_n$ je definován nad abecedou $Z_n = {a_1,a^|_2,\dots,a_n,a^|_n}$ následující gramatikou:
    $S \rightarrow \lambda|SS|a_1Sa^|_1|\dots|a_nSa^|_n.$

    Úvodní pozorování:
    \begin{itemize}
        \item jedá se zřejmě o jazyk bezkontextový
        \item Dyckův jazyk $D_n$ popisuje správně uzávorkované výrazy s $n$ druhy závorek
        \item tímto jazykem lze popisovat výpočty libovolného zásobníkového automatu
        \item pomocí Dyckova jazyka lze popsat libovolný bezkontextový jazyk. 
    \end{itemize}
    \[L = h(D\cap R)\]
    \begin{itemize}
        \item $h$ \dots homomorfismus
        \item $D$ \dots dyckův jazyk
        \item $R$ \dots regulární jazyk
    \end{itemize}
\end{definition}
\subsection{Jak charakterizovat bezkontextové jazyky?}
\begin{itemize}
    \item Pokud do zásobníku pouze přidáváme, stačí si pamatovat poslední symbol; stačí konečná paměť $\rightarrow$ konečný automat
    \item potřebujeme ze zásobníku také odebírat (čtení symbolu), takový proces nelze zaznamenat v konečné struktuře
    \item přidávání a odebírání není zcela libovolné, jedná se o zásobník, t. j. LIFO strukturu
    \item roztáhněme si výpočet se zásobníkem do lineární struktury:
    \begin{itemize}
        \item $X$ symbol přidání
    \end{itemize}
\end{itemize}
\begin{theorem}
    Pro každý bezkontextový jazyk $L$ existuje regulární jazyk $R$ tak, že $L = h(D\cap R)$ pro vhodný Dyckův jazyk $D$ a homomorfismus $h$.
    \begin{proof}
        \begin{itemize}
            \item máme PDA přijímající přijímající $L$ prázdným zásobníkem
            \item převedeme na instrukce tvaru $\delta(q,a,Z)\in (p,w),|w|\leq 2$; delší psaní na zásobník rozdělíme avedením nových stavů
            \item nechť $R^|$ obsahuje všechny výrazy,
            \begin{itemize}
                \item $q^{-1}aa^{-1}Z^{-1}BAp$ pro instrukci $\delta(q,a,Z)\ni (p,AB)$
                \item podobně pro instrukce $\delta(q,a,Z)\in (p,A), \delta(q,a,Z) \in (p,\lambda)$
                \item je-li $a=\lambda$, potom dvojici $aa^{-1}$ nezařazujeme
            \end{itemize}
            \item definujeme $R : Z_0q_0(R^|)^*Q^{-1}$
            \item Dyckův jazyk je definován nad abecedou $\Sigma\cup \Sigma^{-1}\cup Q\cup Q^{-1}\cup \Gamma \cup \Gamma^{-1}$
            \item $D\cap Z_0q_0(R^|)^*Q^{-1}$ popisuje korektní výpočty
            \[Z_0q_0q^{-1}_0aa^{-1}Z^{-1}_0 BApp^{-1}bb^{-1}A^{-1}qq^{-1}cc^{-1}B^{-1}rr^{-1}\]
            \item homomorfismus $h$ vydělí přečtené slovo, t. j.
            \begin{itemize}
                \item $h(a) = a$ pro vstupní (čtené) symboly,
                \item $h(y) = \lambda$ pro ostatní
            \end{itemize}
        \end{itemize}
    \end{proof}
\end{theorem}
\begin{theorem}
    Algoritmus: Konstrukce $PDA$ z $CFG$\\

    Mějme CFG gramatiku $G = (V,T,P,S)$, konstruujeme PDA $P = ({q},T,V\cup T,\delta,q,S)$.
    \begin{enumerate}
        \item Pro neterminály $A \in V, \delta(q,\lambda),A = {(q,\beta)|A \rightarrow \beta \text{ je pravidlo } G}$
        \item pro každý terminál $a \in T, \delta(q,a,a) = {(q,\lambda)}$.
    \end{enumerate}
\end{theorem}

\subsection{Od zásobníkového automatu ke gramatice CFG}
\begin{itemize}
    \item Zásobní automat bere jeden symbol ze zásobníku. Stav před a pro kroku může být různý.
    \item Neterminály gramatiky budou složené symboly $[qXr]$, PDA vyšel z $q$, vzal $X$ a přešel do $r$;
    zavedeme nový počáteční symbol $S$.
\end{itemize}
\begin{theorem}
    Mějme PDA $P = (Q,\Sigma,\Gamma,\delta,q_0,Z_0)$. Pak existuje bezkontextová gramatika $G$ taková, že $L(G) = N(P)$.
    \begin{proof}
        Pravidla definujeme:
        \begin{itemize}
            \item $\forall p \in Q : S \rightarrow [q_0Z_0p]$, t. j. uhodni koncový stav a spusť PDA na $(q_0,w,Z_0) \vdash^* (p,\lambda,\lambda)$.
            \item Pro všechny dvojice $(r,Y_1Y_2 \dots Y_k) \in \delta(q,s,X), s \in \Sigma \cup {\lambda}, \forall r_1,\dots,r_{k-1} \in Q$ vytvoř pravidlo
            \[[qXr_k] \rightarrow s[rY_1r_1][r_1Y_2r_2]\dots[r_{k-1Y_kr_k}]\]
            \item spec. pro $(r,\lambda) \in \delta(q,a,A)$ vytvoř $[qAr] \rightarrow a$.
        \end{itemize}
        Pro $w\in \Sigma^*$ dokazujeme:\\
        $[qXp]\Rightarrow^*w \Leftrightarrow (q,w,X)\vdash^*(p,\lambda,\lambda)$; indukcí v obou směrech (počet kroků PDA, počet kroků derivace).
    \end{proof}
\end{theorem}
\subsection{Greibachové normální forma}
\begin{itemize}
    \item při analýze zhora (tvorbě levé derivace daného slova $w$) potřebujeme vědet, které pravidlo vybrat.
    \item Speciálně vadí pravidla tvaru $A \rightarrow A\alpha$ (levá rekurze).
\end{itemize}
\begin{definition}
    Říkáme, že gramatika je v Greibachové normální formě, jestliže všechna pravidla mají tvar $A \rightarrow a\beta$, kde 
    $a \in T, \beta \in V^*$ (řetězec) \textbf{TODO}
\end{definition}
\begin{definition}
    Jednoznačnost a víceznačnost CFG\\
    \begin{itemize}
        \item Bezkontextová gramatika $G = (V,T,P,S)$ je víceznačná, pokud existuje aspoň jeden řetězec $w\in T^*$ pro který můžeme najíst dva různe derivační stromy
        \item \textbf{TODO}
    \end{itemize} 
\end{definition}
\begin{theorem}
    $L = N(P_{DPDA}) \Rightarrow L$ má jednoznačnou CFG.\\
    \begin{itemize}
        \item Nechť $L=N(P)$ pro nějaký DPDA $P$. Pak $L$ má jednoznačnou CFG
        \item \textbf{TODO}
    \end{itemize}
\end{theorem}
\subsection{Turingovy stroje}
\begin{itemize}
    \item 1931 - 1936 pokusy o formalizaci pojmu algoritmu (Kleene, Church, Turing\dots)
    \item Turingův stroj:
    \begin{itemize}
        \item zachycení práce matematika
        \begin{itemize}
            \item nekonečná tabule, lze z ní číst a lze na ni psát
            \item mozek (řídící jednotka)
        \end{itemize}
        \item Formalizace Turing Machine (TM):
        \begin{itemize}
            \item místo tabule oboustranně nekonečná páska
            \item místo křídy čtecí a zapisovací hlava \textbf{TODO :)}
        \end{itemize}
    \end{itemize}
\end{itemize}
\begin{definition}
    Turingův stroj (TM) je sedmice $M = (Q,\Sigma,\Gamma,\delta,q_0,B,F)$ se složkami:
    \begin{itemize}
        \item $Q$ konečná množina stavů
        \item $\Sigma$ konečná neprázdná množina vstupních symbolů
        \item $\Gamma$ konečná množina všech symbolů pro pásku. Vždy $\Gamma \supseteq \Sigma, Q\cap \Gamma = \emptyset$.
        \item $\delta$ částečná přechodová funkce. $(Q - F) \times \Gamma \rightarrow Q\times \Gamma \times {L,R}$
        \item $\delta(q,x) = (p,Y,D)$:
        \begin{itemize}
            \item $q \in (Q-F)$ aktuální stav
            \item $X \in \Gamma$ aktuální symbol na pásce
            \item $p$ nový stav, $p \in Q$
            \item $Y \in \Gamma$ symbol pro zasání do aktuální buňky, přepíše aktuální obsah
            \item $D \in {L,R}$ je směr pohybu hlavy (doleva, doprava).
        \end{itemize}
        \item $q_0 \in Q$ počáteční stav
        \item $B\in \Gamma - \Sigma$ Blank. Symbol pro prázdné buňky na začátku všude kromě konečného počtu buněk se vstupem
        \item $F \subseteq Q$ množina koncových neboli přijímajících stavů.
    \end{itemize}
\end{definition}
\begin{definition}
    Konfigurace Turingova stroje\\

    - je řetězec $X_1X_2\dots X_{i-1}qX_i X_{i+1}\dots X_n$ kde
    \begin{itemize}
        \item $q$ je stav Turingova stroje
        \item čtecí hlava je vlevo od $i$-tého symbolu
        \item $X_1,\dots,X_n$ je část pásky mezi nejlevějším a nejpravějším symbolem různým od prázdného (B). S výjimkou v případě,
        že je hlava na kraji - pak na tom kraji vkládáme jeden $B$ navíc.
    \end{itemize}
\end{definition}
\begin{definition}
    Krok Turingova stroje $M$ značíme $\vdash_M, \vdash^*_M, \vdash^*$ jako u zásobníkových automatů.
    Pro $\delta(q,X_i) = (p,Y,L)$
    \[X_1X_2\dots X_{i-1}q X_i X_{i+1}\dots X_n \vdash_M X_1 X_2 \dots X_{i-2}pX_{X_i-1} ... \textbf{TODO}\]
\end{definition}
\begin{definition}
    Turingův stroj $M$ přijíma jazyk $L(M) = {w \in \Sigma^*: q_0 w \vdash^*_M \alpha p \beta, p \in F, \alpha, \beta \in \Gamma^*}$, t. j. množinu slov, po jejichž přečtení se dostane do koncového stavu. Pásku (u nás) nemusí uklízet.
    Jazyk nazveme rekurzivně spočetným, pokud je přijíman nějakým Turingovým strojem $T$, t. j. $L = L(T)$. 
\end{definition}
\begin{definition}
    TM zastaví, pokud vstoupí do stavu $q$, s čteným symbolem $X$ a není instrukce pro tuto situaci, t.j. $\delta(q,X)$ není definováno.
\end{definition}
\begin{itemize}
    \item Předpokládáme, že v přijímajícím stavu $q \in F$ TM zastaví
    \item dokud nezastaví, nevíme, jestli přijme nebo nepřijme slovo.
\end{itemize}
\begin{definition}
    Říkáme, že TM $M$ rozhoduje jazyk $L$, pokud $L = L(M)$ a pro každé $w\in \Sigma^*$ stroj nad $w$ zastaví.

    Jazyky rozhodnutelné TM nazýváme $rekurzivní jazyky$.
\end{definition}
\begin{definition}
    Přechodový diagram pro TM sestává z uzlů odpovídajícím stavům TM. Hrany $q \rightarrow p$ jsou označeny seznamem všech dvojic $X/YD$, kde 
    $\delta(q,X) = (p,Y,D), D \in {\leftarrow, \rightarrow}$.
    Pokud neuvedeme jinak, $B$ značí blank - prázdný symbol.
\end{definition}
\end{document}