\documentclass[../main.tex]{subfiles}

%{definition}{Definice}
%{example}{Příklad}
%{intuition}{Intuice}
%{remark}{Poznámka}
%{consequence}{Důsledek}
%{observation}{Pozorování}

\begin{document}
%%%%%%%%%%%%%%%%%%%%%%%%%%%%%%%%%%%%%%%%%%%%%%%%%%%%%%%%%%%%%%%%%%%%%%%%%%%%%%%%
\section{Šestá přednáška}

\begin{definition}[Palindromy]
    Palindrom je řetězec $w$ stejný při čtení zepředu i zezadu, tj. $w = w^R$.
\end{definition}

\begin{theorem}
    Jazyk $L_{pal} = {w : w=w^R, w\in \Sigma^* }$ není regulární.
\end{theorem}
\begin{proof}
    Sporem. Předpokládejme $L_{pal}$ je regulární, nechť $n$ je konstatnta z pumping lemma,
    uvažujme slovo $w = 0^n10^n$.
    
    Z pumping lemmatu lze rozložit $w = xyz$, $y$ obsahuje jednu alebo více z prvních $n$ nul.
    Tedy $xz$ má být v $L_{pal}$ ale není, t.j. není regulární.
\end{proof}

\subsection{Formální (generativní) gramatiky, bezkontextové gramatiky}

\begin{definition}
    Formální (generativní) gramatika je $G = (V,T,P,S)$ složena z
    \begin{enumerate}
        \item konečné množiny \textbf{neterminálů} (variables) $V$,
        \item neprázdné konečné množiny \textbf{terminálních symbolů (terminálů)} T,
        \item \textbf{počáteční symbol} $S \in V$,
        \item konečné množiny \textbf{pravidel (produkcí)} $P$ reprezentující rekurzivní definici jazyka. 
        Každé pravidlo má tvar
        \[\beta A_\gamma \rightarrow A \in V, \beta, \gamma, \omega \in (V \cup B)^*\]
        t. j. levá strana obsahuje aspoň jeden neterminální symbol.
    \end{enumerate}
\end{definition}

\begin{definition}[Bezkontextová gramatika (CFG)]
    je $G = (V,T,P,S)$ gramatika, obsahující pouze pravidla tvaru
    \[A \rightarrow \omega, A \in V, \omega \in (V\cup T)^*\]
\end{definition}

\begin{definition}[Chomského hierarchie]
    \textbf{CHOMSKÉHO HIERARCHIE, DOPLNIT, 112/113}
\end{definition}

\begin{definition}[Derivace $\Rightarrow^*$]
    Mějme gramatiku $G = (V,T,P,S)$.
    \begin{enumerate}
        \item Říkáme, že $\alpha$ se \textbf{přímo přepíše} na $\omega$ (píšeme $\alpha \Rightarrow_{G}\omega$
        nebo $\alpha \Rightarrow \omega$) jestliže 
        \[\exists \beta,\gamma,\eta ,\nu \in (V \cup T)^* : \alpha = \eta \beta \nu, \omega = \eta \gamma \nu \& (\beta \rightarrow \gamma) \in P\]
        \item Říkáme, že $\alpha$ se přepíše na $\omega$ (píšeme $\alpha \Rightarrow^* \omega$) jestliže
        \[\exists \beta_1,\dots \beta_n \in (V \cup T)^* : \alpha = \beta_1 \Rightarrow \beta_2 \Rightarrow \dots \Rightarrow \beta_n = \omega\]
        t. j. $\alpha \Rightarrow^* \alpha$.
        \item posloupnost $\beta_1,\dots , \beta_n$ nazýváme derivací (odvození)
        \item pokud $\forall i \neq j : \beta_i \neq \beta_j$, hovoříme o minimálním odvození
    \end{enumerate}
\end{definition}

\begin{definition}[Jazyk generovaný gramatikou $G$]
    Jazyk $L(G)$ gramatiky $G = (V,T,P,S)$ je množina terminálních řetězců, pro které existuje derivace ze startovního symbolu
    \[L(G) = \{w \in T^* |S \Rightarrow^*_G w\}\]

    Jazyk neterminálu $A \in V$ definujeme $L(A) = \{w \in T^* | A \Rightarrow^*_G w\}$.
\end{definition}

\begin{definition}[Gramatika typu 3, pravá lineární]
    Gramatika $G$ je \textbf{pravá lineární, t.j. Typu 3} pokud obsahuje pouze pravidla tvaru\\

    $A \rightarrow wB,A\rightarrow w, A,B \in V, w \in T^*$.
\end{definition}

\begin{example}
    {\color{white} x}

    \begin{enumerate}
        \item $P = \{S \rightarrow 0S, 1A|\lambda, A\rightarrow 0A|1B, B\rightarrow 0B|1S\}$
        \item $S \Rightarrow 0S \Rightarrow 01A \Rightarrow 011B \Rightarrow 0110B \Rightarrow 01101S \Rightarrow 01101$
    \end{enumerate}
\end{example}

\begin{theorem}
    $L\in RE \Rightarrow L \in \mathcal{L}_3$

    Pro každý jazyk rozpornávaný konečným automatem existuje gramatika typu 3, která ho generuje.
\end{theorem}
\begin{proof}
    \begin{enumerate}
        \item $L = L(A)$ pro deterministický konečný automat $A$.
        \item definujeme gramatiku $G=(Q, \Sigma, P, q_0)$, kde pravidla $P$ mají tvar
        \[p \rightarrow aq \text{ když } \delta(p,a) = q\]
        \[p \rightarrow \lambda \text{ když } p \in F\]
        \item je $L(A) = L(G)$?
        \begin{enumerate}
            \item $\lambda \in L(A) \Leftrightarrow q_0 \in F \Leftrightarrow (q_0 \rightarrow \lambda) \in P
            \Leftrightarrow \lambda \in L(G)$
            \item $a_1 \dots a_n \in L(A) \Leftrightarrow \exists q_0,\dots q_n \in Q : \delta(q_i, a_{i+1}) = q_{i+1},q_n \in F
            \Leftrightarrow (q_0 \implies a_1q_1\implies \dots a_1 \dots a_nq_n \implies a_1,\dots a_n)$ je derivace pro $a_1,\dots a_n
            \Leftrightarrow a_1,\dots a_n \in L(G)$
        \end{enumerate}
    \end{enumerate}
\end{proof}

Příprava převodu gramatiky typu 3 na forall

\begin{enumerate}
    \item Opačný směr
    \begin{enumerate}
        \item pravidla $A \rightarrow aB$ kódujeme do přechodové funkce
        \item pravidla $A \rightarrow \lambda$ určují koncové stavy
        \item pravidla $A \rightarrow a_1,\dots a_n B, A \rightarrow a_1,\dots a_n$ s více neterminály rozepíšeme
        \begin{enumerate}
            \item zavedeme nové neterminály $Y_2,\dots, Y_n, Z_1,\dots, Z_N$
            \item vytvoříme pravidla $A \rightarrow a_1Y_2, Y_2 \rightarrow a_2Y_3,\dots Y_n \rightarrow a_n B$
            \item resp. $Z \rightarrow a_1Z_1, Z_1 \rightarrow a_2Z_2,\dots, Z_{n-1} \rightarrow a_nZ_n, Z_n \rightarrow \lambda$
        \end{enumerate}
        \item pravidla $A \rightarrow B$ obpovídají $\lambda$ přechodům
        \begin{enumerate}
            \item zbavíme se jich tranzitivním uzávěrem
            \item nebo musíme tranzitívně uzavřít $S\rightarrow B$ pro hledání $S \rightarrow \lambda$.
        \end{enumerate}
    \end{enumerate}
\end{enumerate}

\begin{theorem}
    Ke každé gramatice typu 3 existuje gramatika typu 3, která generuje stejný jazyk a 
    obsahuje pouze pravidla ve tvaru: $A \rightarrow aB, A \rightarrow \lambda, A,B \in V, a\in T$.
\end{theorem}
\begin{proof}
    Pro gramatiku $G = (V,T,S,P)$ definujeme $G^| = (V^|,T,S,P^|)$, kde pro každé pravidlo zavedeme dostatečný počet nových
    neterminálů $Y_2,\dots, Y_n, Z_1,\dots, Z_n$ a definujeme
    
    \textbf{DOPLNIT TABULKU}
\end{proof}

\begin{theorem}[$\lambda-NFA$ pro gramatiku typu 3 rozpoznávajíci stejný jazyk]
    Pro každý jazyk $L$ generovaný gramatikou typu 3 existuje $\lambda-NFA$ rozpoznávajíci $L$.
\end{theorem}
\begin{proof}
    \begin{enumerate}
        \item Vezmeme $G = (V,T,P,S)$ obsaující jen pravidla tvaru $A\rightarrow aB, A \rightarrow \lambda$,
        $A,B \in V, a\in T$ generující $L$ (předchozí lemma)
        \item definujeme nedeterministický $\lambda-NFA$ $A = (V,T,\delta, S, F)$ kde :
        \begin{enumerate}
            \item $F = \{A | (A\rightarrow \lambda) \in P\}$
            \item $\delta(A,a) = \{B | (A \rightarrow aB) \in P\}$
        \end{enumerate}
        \item $L(G) = L(A)$
        \begin{enumerate}
            \item \textbf{DOPLNIT, 121}
        \end{enumerate}
    \end{enumerate}
\end{proof}

\begin{definition}[Levé (a pravé) lineární gramatiky]
    Gramatiky typu 3 nazýváme také \textbf{pravé lineární} (neterminál je vždy vpravo).

    Gramatika G je levá lineární, jestliže má pouze pravidla tvaru $A \rightarrow Bw, A \rightarrow w, A,B\in V, w\in T^*$.
\end{definition}

\begin{theorem}
    Jazyky generované levou lineární gramatikou jsou právě regulární jazyky.
\end{theorem}
\begin{proof}
    \textbf{DOPLNIT, 122}
\end{proof}

\begin{definition}[Lineární gramatika, jazyk]
    Gramatika je lineární, jestliže má pouze pravidla tvaru $A \rightarrow uBw, A \rightarrow w, A,B \in V, u, w \in T^*$ (na pravé straně vždy maxiálne jeden neterminál).

    Lineární jazyky jsou právě jazyky generované lineárnimi gramatikami.
\end{definition}

\begin{enumerate}
    \item Zřejmě platí: regulární jazyky $\subseteq$ lineární jazyky,
    \item Jde o vlastní podmnožinu $\subsetneq$.
\end{enumerate}

\begin{example}[Lineární, neregulární jazyk]
    Jazyk $L = \{0^n1^n | n \geq 1\}$ není regulární jazyk, ale je lineární, generovaný gramatikou s pravidly $S \rightarrow 0S1|01$.
\end{example}

\begin{definition}[Bezkontextová gramatika]
    Bezkontextová gramatika je gramatika, kde všechna pravidla jsou tvaru $A \rightarrow \omega, \omega \in (V\cup T)^*$

    CFG pro jednoduché výrazy
    \begin{enumerate}
        \item $E \rightarrow I$
        \item $E \rightarrow E + E$
        \item $E \rightarrow E \ast E$
        \item $E \rightarrow (E)$
        \item $I \rightarrow a$
        \item $I \rightarrow b$
        \item $I \rightarrow Ia$
        \item $I \rightarrow Ib$
        \item $I \rightarrow I0$
        \item $I \rightarrow I1$
    \end{enumerate}

    Pravidla 1 až 4 definují výraz.

    Pravidla 5 až 10 definují identifikátor $I$ odpovídajíci regulárnímu výrazu $(a+b)(a+b+0+1)^*$.
\end{definition}

\begin{definition}[Derivační strom]
    Mějme gramatiku $G = (V,T,P,S)$. Derivační strom pro $G$ je strom, kde 
    \begin{enumerate}
        \item Kořen (kreslíme nahoře) je označen startovním symbolem $S$,
        \item každý vnitřní uzel je ohodnocen neterminálem $V$.
        \item Každý uzel je ohodnocen prvkem $\in V \cup T \cup {\lambda}$.
        \item Je-li uzel ohodnocen $\lambda$, je jediným dítětem svého rodiče.
        \item Je-lin $A$ ohodnocení vrcholu a jeho děti \textit{zleva pořadě} jsou ohodnoceny 
        $X_1,\dots, X_k$, pak $(A \rightarrow X_1, \dots, X_k)\in P$ je pravidlo gramatiky.
    \end{enumerate}
\end{definition}

\begin{definition}[Strom dáva slovo (yield)]
    Říkáme, že derivační strom dáva slovo $w$ (yield), jestliže $w$ je slovo složené z ohodnocení listů bráno zleva doprava.
\end{definition}

\begin{definition}[Levá a pravá derivace]
    \begin{enumerate}
        \item Levá derivace (leftmost) $\Rightarrow_{lm}, \Rightarrow_{lm}^*$ v každém kroku přepisuje nejlevější neterminál.
        \item Pravá derivace (rightmost) $\Rightarrow_{rm}, \Rightarrow_{rm}^*$ v každém kroku přepisuje nejpravější neterminál.
    \end{enumerate}
\end{definition}

\begin{theorem}
    Pro danou gramatiku $G = (V,T,P,S)$ a $w\in T^*$ jsou následujíci tvrzení ekvivalentní:
    \begin{enumerate}
        \item $A \Rightarrow^* w$.
        \item $A \Rightarrow^*_{lm} w$.
        \item $A \Rightarrow^*_{rm} w$.
        \item Existuje derivační strom s kořenem $A$ dávajíci slovo $w$.
    \end{enumerate}
\end{theorem}

\begin{theorem}
    Mějme $CFG G = (V,T,P,S)$ a derivační strom s kořenem $A$ dávající slovo $w \in T^*$. Pak existuje
    levá derivace $A \Rightarrow^*_{lm} w $ v $G$.
\end{theorem}

\begin{lemma}[Příprava důkazu, \uv{obalení derivace}]
    Mějme následující derivaci:
    \[E \Rightarrow I \Rightarrow Ib \rightarrow ab.\]
\end{lemma}
\begin{proof}
    Pro libovolná slova $\alpha, \beta \in (V\cup T)^*$ je také derivace:
    \[\alpha E \beta \Rightarrow \alpha I \beta \Rightarrow \alpha Ib \beta \Rightarrow \alpha ab \beta.\]
\end{proof}

\begin{proof}
    $\exists$ derivační strom, pak $\exists$ levá derivace $\Rightarrow_{lm}$\\

    Indukcí podle výšky stromu:
    \begin{enumerate}
        \item Základ: výška 1: Kořen $A$ s dětmi dávajícimi $w$. Je to derivační strom, proto
        $A \rightarrow w$ je pravidlo $\in P$, tedy $A \Rightarrow_{lm} w$ v jednom kroku.
        \item Indukce: výška $n > 1$. Kořen $A$ s dětmi $X_1,X_2,\dots,X_k$.
        \begin{enumerate}
            \item Je-li $X_i \in T$, definujeme $w_i \equiv X_i$.
            \item Je-li $X_i \in V$, z indukčního předpokladu $X_i \Rightarrow^*_{lm}w_i$.
        \end{enumerate}

        Levou derivaci konstruujeme induktivně pro $i = 1,\dots,k$\\
        složíme $A \Rightarrow^*_{lm}w_1w_2 \dots w_i X_{i+1}X_{i+2}\dots X_k$.
        \begin{enumerate}
            \item Pro $X_i \in T$ jen zavedeme čítač $i + +$.
            \item Pro $X_i \in V$ přepíšeme derivaci: $X_i\Rightarrow_{lm} \alpha_1 \Rightarrow_{lm} \alpha_2\dots \Rightarrow_{lm}w_i na$
            \[w_1w_2\dots w_{i-1}X_iX_{i+1}X_{i+2}\dots X_k \Rightarrow_{lm}\]
            \[w_1w_2\dots w_{i-1}\alpha_1X_{i+1}X_{i+2}\dots X_k \Rightarrow_{lm}\]


            \[\Rightarrow_{lm} w_1w_2\dots w_{i-1}w_i X_{i+1}X_{i+2}\dots X_k\]                
        \end{enumerate}

        Pro $i = k$ dostaneme levou derivaci $w \in A$.
    \end{enumerate}
\end{proof}
\end{document}