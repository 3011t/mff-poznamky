\documentclass[../main.tex]{subfiles}

%{definition}{Definice}
%{example}{Příklad}
%{intuition}{Intuice}
%{remark}{Poznámka}
%{consequence}{Důsledek}
%{observation}{Pozorování}

\begin{document}
%%%%%%%%%%%%%%%%%%%%%%%%%%%%%%%%%%%%%%%%%%%%%%%%%%%%%%%%%%%%%%%%%%%%%%%%%%%%%%%%
\section{Štrnáctá přednáška}

\begin{definition}
    Rozhodnutelný problém\\

    Problémem $P$ mysl9me matematicky/informaticky definovanou množinu otázek kódovatelnou řetězci nad abecedou $\Sigma$ s odpověďmi $\in {\text{ano, ne}}$

    Problém je (algoritmicky) rozhodnutelný, pokud existuje Turingův stroj TM takový, že pro každý vstup $w \in P$ zastaví a navíc přijíme právě když $P(w) = \text{ano}$
    (t. j. pro $P(w) = \text{ne}$ zastaví v ne-přijímajícím stavu).\\
    Problém, který není algoritmicky rozhodnutelný nazýváme nerozhodnutelný problém. 
\end{definition}
\begin{example}
    \begin{itemize}
        \item Obsahuje vstupní slovo pět nul?
        \item Je vstupní slovo korektně definovaným kódem Turingova stroje v kódování výše?
        \item \textbf{DOPLNIT}
    \end{itemize}
\end{example}

\begin{definition}
    Redukce\\
    Redukci problému $P_1$ na $P_2$ nazýváme algoritmus $R$, který pro každou instanci $w\in P_1$ zastaví a vydá $R(w) \in P_2$ tak, že
    \begin{itemize}
        \item $P_1(w) = \text{ano}$ právě kdydž $P_2(R(w)) = \text{ano}$
        \item t. j. i $P_1(w) = \text{ne}$ právě když $P_2(R(w)) = \text{ne}.$
    \end{itemize}
\end{definition}
\begin{theorem}
    Redukce\\
    Pokud existuje redukce problému $P_1$ na $P_2$, pak:
    \begin{itemize}
        \item Pokud $P_1$ je nerozhodnutelný, pak je nerozhodnutelný i $P_2$.
        \item Pokud $P_1$ není rekurzivně spočetný, pak není $RE$ ani $P_2$.
    \end{itemize}
    \begin{proof}
        \begin{itemize}
            \item Předpokládejme $P_1$ nerozhodnutelný, Je-li možné rozhodnout $P_2$, pak můžeme zkombinovat redukci $P_1$ na $P_2$ s algoritmem
            rozhodujícím $P_2$ pro konstrukci algoritmu rozhodujícího $P_1$. Proto je $P_2$ nerozhodnutelný.
            \item Předpokládejme $P_1$ ne-RE, ale $P_2$ je RE. Podobně jako výše zkombinujeme redukci a výsledek $P_2$ k důkazu $P_1$ je RE, spor.
        \end{itemize}
    \end{proof}
\end{theorem}
\begin{theorem}
    Nerozhodnutelnost univerzálního jazyka\\
    $L_u$ je rekurzivně spočetný, ale není rekurzivní.
\end{theorem}
\begin{theorem}
    Problem zastavení\\

    Instanci problému zastavení je dvojice řetězců $M,w \in {0,1}*.$ Hledáme algoritmus $Halt(M,w)$, který vydá 1 právě když stroj $M$ zastaví na vstupu $w$, jinak vydá 0.
    Problém zastavení je nerozhodnutelný.
    \begin{proof}
        \begin{itemize}
            \item Redukujeme $L_d$ na $Halt$.
            \item Pře... \textbf{TODO}
        \end{itemize}
    \end{proof}
\end{theorem}
Tu bol jeden slide ktorý som absolutne nestihol

\begin{definition}
    Postův korespondenční problém\\

    Instance Postova korespondenčního problému (PCP) jsou dva seznamy slov nad abecedou $\Sigma$ značené $A = w_1,w_2,\dots, w_k$ a $B = x_1,\dots,x_k$ stejné délky $k$.
    Pro každé $i$ dvojice $(w_i,x_i)$ se nazývá odpovídající dvojice.
    Instance PCP má řešení, pokud existuje posloupnost jednoho či více přirozených čísel $i_1,\dots,i_m$ tak, že $w_{i_1},\dots,w_{i_m} =x_{i_1},\dots,x_{i_m}$, t.j. \dots nestihol som
\end{definition}
\begin{example}
    TODO
\end{example}
\begin{definition}
    Částečné řešení\\
    Částečným řešením nazýváme posloupnost indexů $i_1,\dots,i_r$ taková, že jeden z řetězců $w_{i_1},\dots,w_{i_r}$ a $x_{i_1},\dots,x_{i_r}$ je prefix druhého (i v případě, že řetězce nejsou totožné).
\end{definition}
\begin{theorem}
    Je-li posloupnost čísel řešením, pak je každý prefix částečným řešením.
\end{theorem}
\begin{definition}
    Modifikovaný PCP (MPCP)\\
    Mějme PCP, t. j. seznamy $A = w_1,\dots,w_k, B = x_1,\dots,x_k$. Hledáme seznam 0 nebo více přirozených čísel $i_1,i_2,\dots,i_m : w_1,w_{i_1},w_{i_2},\dots,w_{i_m} = x_1,x_{i_1},x_{i_2},\dots,x_{i_m}$. V tom případě
    říkáme, že $PCP$ má iniciální řešení. Modifikovaný Postův korespondenční problém: má PCP iniciální řešení?
\end{definition}
\begin{example}
    \textbf{TODO}
\end{example}
\begin{theorem}
    Redukce MPCP na PCP\\
    $w\in MPCP$ má iniciální řešení právě když má $R(w)$ řešení.
    \begin{proof}
        \begin{itemize}
            \item Vezměme nové symboly $\ast, \$ \neq \Sigma$.
            \item $\forall i = 1,\dots,k$ definujeme $y_i$ rozšířením $w_i$ s $\ast$ za každým písmenem $w_i$
            \item $\forall i = 1,\dots,k$ definujeme $z_i$ rozšířením $x_i$ s $\ast$ před každým písmenem $x_i$.
            \item $y_0 = \ast y_1, z_0 = z_1$.
            \item $y_{k+1} = \$, z_{k+1} = \ast \$$.
            \item $i_1,i_2,\dots,i_m$ je iniciální řešení $\Leftrightarrow$ $0,i_1,i_2,\dots,i_m, (k+1)$ je řešení PCP.
        \end{itemize}
    \end{proof}
    \begin{example}
        \textbf{TODO}
    \end{example}
\end{theorem}
\begin{itemize}
    \item Chceme dokázat, žet PCP je algoritmicky nerozhodnutelný.
    \item Redukovali jsme MPCP na PCP a redukujeme $L_u$ na MPCP. Chceme dokázat:
    \begin{itemize}
        \item $\Rightarrow$ Pokud $w\in L(M)$, začneme iniciálním párem a simulujeme výpočet $M$ na $w$.
        \item $\Leftarrow$ Máme-li iniciální řešení PCP, odpovídá přijímajícímu výpočtu $M$ nad $w$.
        \begin{itemize}
            \item MPCP musí začít první dvojicí
            \item \textbf{TODO}
        \end{itemize}
    \end{itemize}
\end{itemize}
\begin{algorithm}
    Redukce $L_u$ na MPCP\\

    Konstruujeme MPCP pro TM $M = (Q,\Sigma,\Gamma, \delta, q_0, B,F)$, který nikdy nepíše $B$ a nejde hlavou doleva od počáteční pozice. Nechť $w\in \Sigma*$ je vstupní slovo.
    
    \textbf{TODO tabulka}
\end{algorithm}
Pro bezkontextové jazyky je algoritmicky rozhodnutelné
\begin{itemize}
    \item zda dané slovo patří či nepatří do jazyka
    \begin{itemize}
        \item prázdné slovo zvlášť
        \item pak algoritmus CYK
        \item nebo otestovat všechny derivace s $2|w| - 1$ pravidly
    \end{itemize}
    \item zda je jazyk prázdný
    \begin{itemize}
        \item algoritmus redukce gramatiky (ne-generujících a nedosažitelných), zjistíme, zda lze z $S$ generovat terminální slovo.
    \end{itemize}
\end{itemize}
\begin{theorem}
    Je algoritmicky nerozhodnutelné, zda je bezkontextová gramatika víceznačná.
    \begin{proof}
        Mějme instanci PCP ($A = w_1,w_2,\dots,w_k, B = x_1,x_2,\dots,x_k$) množinu indexů $a_1,a_2,\dots,a_k \in N$ a tři gramatiky $G_A,G_B,G_C$:
        
        \textbf{TODO}
    \end{proof}
\end{theorem}

\begin{theorem}
    Mějme bezkontextové gramatiky $G_1,G_2$ a $R$ regulární výraz. Následující problémy jsou algoritmicky nerozhodnutelné:
    \begin{enumerate}
        \item Je $L(G_1) \cap L(G_2) = \emptyset$ ?
        \item Je $L(G_1) = T*$ pro nějakou abecedou $T$ ?
        \item Je $L(G_1) = L(G_2) $ ?
        \item Je $L(G_1) = L(R)$ ?
        \item Je $L(G_1) \subseteq L(G_2) $ ?
        \item Je $L(R) \subseteq L(G_2)$ ?
    \end{enumerate} 
    \begin{remark}
        Důkazy jsou v slidech
    \end{remark}
\end{theorem}
\subsection{Shrnutí}
Popis nekonečných objektů konečnými prostředky
\begin{itemize}
    \item regulární jazyky
    \begin{itemize}
        \item konečné automaty (NRA, 2FA)
        \item Nerode (rozklad), Kleene (elementární operace), pumpování
    \end{itemize}
    \item bezkontextové jazyky
    \begin{itemize}
        \item zásobníkové automaty (DPDA $\neq$ PDA)
        \item pumpování
    \end{itemize}
    \item kontextové jazyky
    \begin{itemize}
        \item lineárně omezené automaty
        \item monotonie
    \end{itemize}
    \item rekurzivně spočetné jazyky
    \begin{itemize}
        \item Turingovy stroje
        \item algoritmická nerozhodnutelnost
    \end{itemize}
\end{itemize}
\end{document}
