\documentclass[../main.tex]{subfiles}

%{definition}{Definice}
%{example}{Příklad}
%{intuition}{Intuice}
%{remark}{Poznámka}
%{consequence}{Důsledek}
%{observation}{Pozorování}

\begin{document}
%%%%%%%%%%%%%%%%%%%%%%%%%%%%%%%%%%%%%%%%%%%%%%%%%%%%%%%%%%%%%%%%%%%%%%%%%%%%%%%%
\section{Devátá přednáška}

\begin{definition}
    Chomského normální tvar\\

    O bezkontextové gramatice $G=(V,T,P,S)$ bez zbytečných symbolů kde jsou všechna pravidla v jednou ze dvou tvarů
    \begin{enumerate}
        \item $A \righarrow BC, A,B,C\in V,$
        \item $A \rightarrow a, A \in V, a \in T$,
    \end{enumerate}
    říkáme, že je v Chomského normálním tvaru (ChNF).

    Potřebujeme dva další kroky:
    \begin{enumerate}
        \item pravé strany délky 2 a více předělat na samé neterminály
        \item rozdělit pravé strany délky 3 a více neterminálů na více pravidel
    \end{enumerate}

    \begin{remark}
        neterminály: 
        \begin{enumerate}
            \item pro každý terminál $a$ vytvoříme nový neterminál, řekněme $A$.
            \item přidáme pravidlo $A \rightarrow a$,
            \item použijeme $A$ místo $a$ na pravé straně pravidel délky 2 a více.
        \end{enumerate}
    \end{remark}
    \begin{remark}
        
    \end{remark}
\end{definition}

\begin{theorem}
    Gramatika v normálním tvaru, redukovaná\\

    Mějme bezkontextovou gramatiku $G, L(G) - {\lambda} \neq \emptyset$. Pak existuje $CFG G_1 : L(G_1) = L(G) - {\lambda}$ a 
    $G_1$ neobsahuje $\lambda$-pravidla, jednotková pravidla ani zbytečné symboly. Gramatika $G_1$ se nazývá redukovaná.
    \begin{proof}
        \begin{enumerate}
            \item Začneme eliminací $\lambda$-pravidel
            \item Eliminujeme jednotková pravidla. Tím nepřidáme $\lambda$-pravidla.
            \item Eliminujeme zbytečné symboly. Tím nepřidáme žádná pravidla.
        \end{enumerate}
    \end{proof} 
\end{theorem}

\begin{theorem}
    ChNF\\

    Mějme bezkontextovou gramatiku $G, L(G) - {\lambda} \neq \emptyset$. Pak existuje $CFG G_1$ v ChNF taková, že $L(G_1) = L(G) - {\lambda}$
\end{theorem}
\begin{example}
    \textbf{DOPLNIT Z DISCORDU 8.10}
\end{example}

\begin{theorem}
    Velikost derivačního stromu gramatiky v $CNF$\\

    Mějme derivační strom podle gramatiky $G = (V,T,P,S)$ v ChNF, který dáva slovo $w$. Je-li délka nejdelší cesty $n$,
    pak $|w| \leq 2^{n-1}$.
    \begin{proof}
        
        Indukcí dle $n$,
        \begin{enumerate}
            \item $|a| = 1 = 2^0$
            \item $2^{n-2} + 2^{n-2} = 2^{n-1}$.
        \end{enumerate}
    \end{proof}
    \begin{remark}
        Mějme derivační strom podle gramatiky $G = (V,T,P,S)$ v ChNF, který dáva slovo $w, |w| > p = 2^{n-1}$. Pak ve stromě existuje cesta delší než $n$.
    \end{remark}
\end{theorem}

\begin{theorem}
    Lemma o vkládání (pumping) pro bezkontextové jazyky\\

    Mějme bezkontextový jazyk $L$. Pak $ \exists n \in \mathbb{N} : \forall z\in L, |z| > n : z = uvwxy :$
    \begin{enumerate}
        \item $|vwx| \leq n$
        \item $vx \neq \lambda$
        \item $\forall i \geq 0, uv^iwx^iy \in L$.
    \end{enumerate}
    \textbf{OBRAZOK 8.2}
    \begin{proof}
        \begin{enumerate}
            náznak důkazu:
            \item vezmeme derivační strom pro z
            \item najdeme nejdelší cestu
            \item na ní dva stejné neterminály
            \item tyto neterminály určí dva podstromy.
            \item podstromy definují rozklad slova
            \item nyní můžeme větší podstrom posunout (i > 1)
            \item nebo nahradit menším podstromem (i = 0)
        \end{enumerate}
        Důkaz:
        \begin{enumerate}
            \item vezmeme gramatiku v ChNF ( pro $L = {\lambda}$ a $\emptyset$ zvolíme   $n = 1$)
            \item Nechť $|V| = k; n=2^k$.
            \item Pro $z\in L, |z| > 2^k$ má v derivačním stromu z cestu delší než $k$.
            \item vezmeme nejdelší cestu; terminál kam vede označíme $t$.
            \item ASpoň dva z posledních $(k+1)$ neterminálů na cestě do $t$ jsou stejné
            \item vezmeme dvojici $A^1,A^2$ nejblíže k $t$ (určuje podstromy $T^1,T^2$)
            \item cesta z $A^1$ do $t$ je nejdelší v podstromu $T^1$ a má délku maximálně $(k+1)$ (tedy slovo dané stromem $T^1$ není delší než $2^k$ (tedy $|vwx| \leq n$))
            \item z $A^1$ vedou vě cesty (ChNF), jedna do $T^2$, druhá do zbytku $wx$ (ChNF je nevyouštějící, tedy $vx \neq \lambda$)
            \item derivace slova ($A^1 \Rightarrow^* vA^2x, A^2 \Rightarrow^* w$); $S \Rightarrow^* uA^1y \Rightarrow^* uvA^2xy \Rightarrow^* uvwxy$
            \item posuneme-li $A^2$ do $A^1$ $\rightarrow S \Rightarrow^* uA^2y \Rightarrow^* uwy$
            \item posuneme-li $A^1$ do $A^2$ ($i = 2,3,\dots$) $righarrow S \Rightarrow^* uA^1y \Rightarrow^* uvA^1xy \Rightarrow^* uvvA^2xxy \Rightarrow^* uvvwxxy$. 
        \end{enumerate}
    \end{proof}
\end{theorem}
\begin{definition}
    Deterministický zásobníkový automat (DPDFA)\\

    Zásobníkový automat $P = (Q,\Sigma, \Gamma, \delta, q_0,Z_0,F)$ je deterministický PDA právě když platí:
    \begin{enumerate}
        \item $\delta(q,a,X)$ je nejvýše jednoprvková $\forall (q,a,X) \in Q \times (\Sigma \cup {\lambda})\times \Gamma$.
        \item Je-li $\delta(q,a,X)$ neprázdná pro nějaké $a \in \Sigma$, pak $\delta(q,\lambda,X)$ musí být prázdná.
    \end{enumerate}
\end{definition}

\begin{theorem}
    \[RL \subsetneq L(P_{DPDA}) \subsetneq L(P_{PDA}) = CFL = N(P_{PDA}) \supsetneq N(P_{DPDA})\]

    Nechť $L$ je regulární jazyk, pak $L = L(P)$ pro nějaký $DPDA P$.
    \begin{proof}
        DPDA může simulovat deterministický konečný automat a ignorovat zásobník (nechat tam $Z_0$).
    \end{proof}
    \begin{remark}
        Jazyk $L_{wcwr}$ je přijímaný DPDA ale není regulární.\\

        Důkaz neregularity z pumping lemmatu na slovo $0^nc0^n$.
    \end{remark}
\end{theorem}
\end{document}