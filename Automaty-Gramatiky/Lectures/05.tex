\documentclass[../main.tex]{subfiles}

%{definition}{Definice}
%{example}{Příklad}
%{intuition}{Intuice}
%{remark}{Poznámka}
%{consequence}{Důsledek}
%{observation}{Pozorování}

\begin{document}
%%%%%%%%%%%%%%%%%%%%%%%%%%%%%%%%%%%%%%%%%%%%%%%%%%%%%%%%%%%%%%%%%%%%%%%%%%%%%%%%
\section{Pátá přednáška}

\begin{example}
    Nechť $A = (Q,\Sigma, \delta, q_0, F)$ je DFA. Definujeme jazyk $L = \{w \in \Sigma^*; \delta^*(q_0,w)\in F\}$
    a pro každý stav $q\in Q$ existuje prefix $x_q$ slova $w$ tak, že $\delta^*(q_0,x_q) = q$.

    Tento jazyk $L$ je regulární.

    \begin{enumerate}
        \item $M$ označme $M = L(A)$.
        \item $T$ definujeme novou abecedu $T$ trojic $\{[paq]; p,q \in Q, a \in \Sigma, \delta(p,a) = q\}$.
        \item $h$ definujeme homomorfizmus $(\forall p,q,a) h([paq]) = a$.
        \item $L_1$ Jazyk $L_ = h^{-1}(M)$ je regulární, protože $M$ je regulární (DFA inverzní homomorfizmus).
        \item $h^{-1}(101)$ obsahuje $2^3 = 8$ řetězců, např.
         \[[p1p][q0q][p1p] \in {[p1p],[q1q]}{[p0q],[q0q]}{[p1p],[q1q]}.\]
    \end{enumerate}
    $L_2$ Vynutíme začátek $q_0$. Definujeme:

    $E_1 = \bigcup_{a\in \Sigma, q\in Q}{[q_0aq]} = $\\
    $E_1 = {[q_0 a_1 q_0],[q_0 a_2 q_1],\dots,[q_0 a_m q_n]}$\\
    Pak $L_2 = L_1 \cap L(E_1,T^*)$.

    $L_3$ Vynutíme stejné sousedíci stavy. Definujeme ne-odpovídajíci dvojice:

    $E_2 = \bigcup_{q\neq r, p,q,r,s \in Q, a,b \in \Sigma}{[paq][rbs]}.$\\
    Definujeme $L_3 = L_2 - L(T^*.E_2.T*)$.

    Končí v přijímajícím stavu, protože jsme začali v jazyku $M$ přijímaném DFA $A$.
    
    $L_4$ Všechny stavy. $\forall q\in Q$ definujeme $E_q$ jako regulární výraz sjednocení všech symbolů
    $T$ takových, že $q$ není ani na první, ani na poslední pozici. Odečteme $L(E^*_q)$ od $L_3 . L_4 = L_3 - \bigcup_{q\in Q}{E^*_q}$.

    $L$ Odstráníme stavy, necháme symboly. $L = h(L_4)$, tedy $L$ je regulární.
\end{example}
\begin{theorem}
    Lze algoritmicky rozhodnou, zda jazyk přijímaný $DFA,NFA,\lambda -NFA$ je prázdný.\\
    Jazyk je prázdný právě když žádny z koncových stavů není dosažitelný. Dosažitelnost lze testovat $O(n^2)$.
\end{theorem}
\begin{theorem}
    Pro daný řetězec $w: |w| = n$ a regulární jazyk $L$ lze algoritmicky rozhodnout, zda je $w\in L$.
    \begin{proof}
        \begin{enumerate}
            \item DFA: Spusť automat, pokud $|w| = n$, při dobré reprezentaci a konstantním čase přechodu $O(n)$.
            \item NFA o $s$ stavech: čas $O(ns^2)$. Každý vstupní symbol aplikujeme na všechny 
            \item \textbf{DOPLNIT, NESTIHAM}
        \end{enumerate}
    \end{proof}
\end{theorem}

\begin{definition}
    algebraický popis jazyků\\

    Pro konečnou neprázdnou abecedu $\Sigma$ označme $RJ(\Sigma)$ nejmenší třídu jazyků, která:
    \begin{enumerate}
        \item obsahuje prázdný jazyk $\emptyset$,
        \item pro každé ... 
        \item \textbf{NESTIHAM}
    \end{enumerate}
\end{definition}
\begin{theorem}
    Kleene
    
    \textbf{NESTIHAM}
\end{theorem}

Dvousměrné (dvoucestné) konečné automaty

Konečný automat provádí následující činnosti:
\begin{enumerate}
    \item přečte písmeno
    \item změní stav vnitřní jednotky
    \item posune čtecí hlavu doprava
\end{enumerate}

Čtecí hlava se nesmí vracet.

\begin{definition}
    Dvousměrným (dvoucestným) konečným automatem nazýváme pětici $A = (Q,\Sigma, \delta, q_0,F)$, kde
    \begin{enumerate}
        \item $Q$ je konečná množina stavů,
        \item $\Sigma$ je konečná množina vstupních symbolů přechodové funkce $\delta$ je zobrazení $Q \times \Sigma \rightarrow Q \times {-1,1}$ rozšířené o pohyb hlavy
        \item $q_0 \in Q$ počáteční stav,
        \item množina přijímajícich stavů $F \subseteq Q$.
    \end{enumerate}
    \begin{remark}
        Je deterministický, nedeterministický zavádět nebudeme.
    \end{remark}
    \begin{remark}
        Nulový pohyb hlavy lze, jen trochu zkomplikuje důkaz dále.
    \end{remark}
\end{definition}
\begin{definition}
    Slovo $w$ je přijato dvousměrným konečným automatem, pokud:
    \begin{enumerate}
        \item výpočet začal na prvním písmenu slova $w$ vlevo v počátečním stavu,
        \item čtecí hlava poprvé opustila slovo $w$ vpravo v některém přijímajícim stavu,
        \item mimo čtené slovo není výpočet definován (výpočet zde končí a slovo není přijato).
    \end{enumerate}
    \begin{remark}
        \begin{enumerate}
            \item Ke slovům si můžeme přidat speciální koncové znaky $\# \notin \Sigma$
            \item funkce $\partial_{\#}$ odstrání $\#$ zleva, $\partial^R_{\#}$ zprava.
            \item Je-li $L(A) = {\#w\#|w\in L \subseteq \Sigma^*}$ regulární, potom i $L$ je regulární
            \item $L = \partial_{\#}$... \textbf{DOPLNIT}
        \end{enumerate}
    \end{remark}
\end{definition}

\begin{theorem}
    Jazyky přijímané dvousměrným konečným automatem jsou právě regulární jazyky.
    \begin{proof}
        konečný automat $\rightarrow$ dvousměrný automat
        
        \textbf{NESTIHOL SOM}
    \end{proof} 
\end{theorem}
\begin{theorem}
    Algoritmus: Funkce $f_u$ popisujíci výpočet $2DFA$ nad slovem $u$\\

    Definujeme funkci $f_u : Q \cup {q^|_0} \rightarrow Q\cup {0}$
    \begin{enumerate}
        \item $f_u(q^|_0)$ popisuje v jakém stavu poprvé odejdeme vpravo, pokud začneme výpočet vlevo v počátečním stavu $q_0$,
        \item $f_u(p): p \in Q$ v jakém stavu opět odejdeme vpravo, pokud začneme výpočet vpravo v $p$,
        \item symbol $0$ značí, že daná situace nenastane (odejdeme vlevo nebo cyklus)
        \item Definujeme ekvivalenci slov následovně: $u \sim w \Leftrightarrow_def f_u = f_w$
        \begin{remark}
            t.j. slova jsou ekvivalentní, pokud mají stejné 'výpočtové' funkce
        \end{remark}
    \end{enumerate}

    Regulárnost 2DFA\\

    \textbf{NESTIHOL SOM}
    
    \begin{proof}
        \begin{enumerate}
            \item Potřebujeme převést návraty na lineární výpočet
            \item Zajímají nás jen přijímajíci výpočty
            \item Díváme se na řezy mezi symboly (v jakém stavu přechází na další políčko)
        \end{enumerate}        

        Pozorování:
        \begin{enumerate}
            \item stavy se v přechodu řezu střídají (doprava, doleva)
            \item první stav jde doprava, poslední také doprava
            \item v deterministických přijímajících výpočtech nejsou cykly
            \item první a poslední řez obsahují jediný stav
        \end{enumerate}

        Formální převod 2DFA na NFA\\

        Nechť $A = (Q,\Sigma, \delta, q_0,F)$ je dvousměrný (deterministický) konečný automat.
        Definujeme ekvivalentní nedeterministický automat $B = (Q^|,\Sigma,\delta^|,(q_0),F^|)$, kde:
        \begin{enumerate}
            \item $Q^|$ jsou všechny korektní přechodové posloupnosti
            \begin{enumerate}
                \item posloupnosti stavů $(q^1,\dots,q^k): q^i \in Q$
                \item délka posloupnosti je lichá $(k = 2m+1)$
                \item žádný stav se neopakuje na liché ani sudé pozici $(\forall i \neq j) (q^{2i} \neq q^{2j})
                \& (\forall i \neq j)(q^{2i+1}\neq q^{2j+1})$
            \end{enumerate}
            \item $F^| = {(q)|q \in F}$ posloupnosti délky 1
            \item $\delta^|(c,a) = \{d|d\in Q^| \& c\rightarrow^a  d $ je lokálně konzistentní přechod pro a $\}$
            \begin{enumerate}
                \item existuje bijekce : $h : c_{odd}\cup d_{even} \rightarrow c_{even}\cup d_{odd}$, tak, že:
                \item pro $h(q) \in c_{even}$ je $(h(q),-1) = \delta(q,a)$
                \item pro $h(q) \in d_{odd}$ je $(h(q),+1) = \delta(q,a)$
            \end{enumerate}
        \end{enumerate}
    \end{proof} 
\end{theorem}

Automaty s výstupem (motivace)\\

\begin{enumerate}
    \item Dosud jediná zpráva z automatu: 'Jsme v přijímajícim stavu'.
    \item Můžeme z FA získat více informací? Můžeme ... \textbf{NESTIHAM}
\end{enumerate}

\begin{definition}
    Mooreův stroj
    
    \textbf{NESTIHOL SOM}
\end{definition}
\begin{example}
    Mooreův stroj pro tenis\\

    Mooreův stroj pro počítaní tenisového skóre.
    \begin{enumerate}
        \item Vstupní abeceda: ID hráče, který uhrál bod
        \item Výstupní abeceda $\&$ stavy: skóre (t. j. $Q = Y, \mu(q) = q$
    \end{enumerate}
    \textbf{Doplniť obrázok z prednášok (98. slide)}
\end{example}

\begin{definition}
    Mealyho stroj\\

    Mealyho (sekvenčním) strojem nazýváme šestici $A = (Q,\Sigma, Y, \delta, \lambda_M, q_0 )$, resp.
    pětici $A = (Q,\Sigma, Y, \delta, \lambda_M)$, kde
    \begin{enumerate}
        \item Q je konečná neprázdná množina stavů
        \item $\Sigma$ je konečná neprázdná množina symbolů (vstupní abeceda)
        \item $Y$ je konečná neprázdná množina symbolů (výstupní abeceda)
        \item \textbf{NESTIHAM}
    \end{enumerate}
    \textbf{TU BOL TIEZ NEJAKY TEXT ESTE}
\end{definition}
\begin{example}
    Mealyho stroj - automat, který dělí vstupní slovo v binárním tvaru číslem 8 (celočíselně).
    \begin{enumerate}
        \item Posun o tři bity doprava
        \item potřebujeme si pamatovat poslední trojici bitů
        \item vlastně tříbitová dynamická paměť
    \end{enumerate}
    I když nevíme, kde automat startuje, po třech symbolech začne počítat správně.
\end{example}

\begin{theorem}
    Převod Mealyho stroje na Mooreův\\

    Pro každý Mealyho stroj existuje Mooreův stroj převádějící každé vstupní slovo na stejné výstupní slovo.
   
    \begin{proof}
        Sestrojme Mooreův stroj $B$ tak, aby $\forall q,w$ \textbf{NESTIHOL SOM :)}
    \end{proof}
\end{theorem}
\end{document}