\documentclass[../main.tex]{subfiles}

%{definition}{Definice}
%{example}{Příklad}
%{intuition}{Intuice}
%{remark}{Poznámka}
%{consequence}{Důsledek}
%{observation}{Pozorování}

\begin{document}
%%%%%%%%%%%%%%%%%%%%%%%%%%%%%%%%%%%%%%%%%%%%%%%%%%%%%%%%%%%%%%%%%%%%%%%%%%%%%%%%
\section{Dvanáctá přednáška}
\begin{definition}
    Separovaná gramatika\\
    Gramatika je separovaná, pokud obsahuje pouze pravidla tvaru $\alpha \rightarrow \beta$, kde:
    \begin{itemize}
        \item buď $\alpha,\beta \in V^+$ (neprázdné posloupnosti neterminálů)
        \item nebo $\alpha \in V; \beta \in T \cup {\lambda}$.
    \end{itemize}
\end{definition}
\begin{theorem}
    Tu bolo nejaké lemma ktoré som nestihol napísať
    \begin{proof}
        a tu bola proof ktorú som nestihol napísať
    \end{proof}
\end{theorem}
\begin{definition}
    monotónní (nevypouštějící) gramatika\\
    Gramatika je monotónní, jestliže pro každé pravidlo $(\alpha \rightarrow \beta)\in P : |\alpha| \leq |\beta|$.  
\end{definition}
\begin{theorem}
    Ke každé monotónní gramatice lze nalézt ekvivalentní gramatiku kontextovou
    \begin{proof}
        \begin{itemize}
            \item nejprve převedeme gramatiku na separovanou - tím se monotonie neporuší (a pravidla $A'\rightarrow a$ jsou kontextová)
            \item zbývající pravidla $A_1\dots A_m \rightarrow B_1 \dots B_n$, $m \leq n$ převedeme na pravidla s novými terminály $C$
        \end{itemize}
        \textbf{TODO, tabulka, pod definicí 12.2, 238/234-314}
    \end{proof}
\end{theorem}
\subsection{Lineárně omezené automaty}
Ještě potřebujeme ekvivalent pro kontextové gramatiky. Kontextovou gramatiku dostaneme z libovolné monotónní gramatiky.
\begin{definition}
    Lineárně omezený automat LBA je nedeterministický TM, kde na pásce je označen levý a pravý konec, $l,r$. Tyto symboly nelze
    při výpočtu přepsat a nesmí se jít nalevo od $l$ a napravo od $r$.\\
    Slovo $w$ je přijímano lineárně omezeným automatem, pokud $q_0 l w r \vdash^*$\textbf{TODO}
\end{definition}
\begin{theorem}
    Každý kontextový jazyk lze přijímat pomocí LBA.
    \begin{proof}
        \begin{itemize}
            \item derivaci gramatiky budeme simulovat pomocí LBA.
            \item použijeme pásku se dvěma stopami
            \item slovo $w$ dáme nahoru, na začátek dolní stopy $S$
            \item přepisujeme slovo ve druhé stopě podle pravidel $G$
            \begin{itemize}
                \item nedeterministicky vybereme část k přepsání
                \item provedeme přepsání dle pravidla (pravá část se odsune)
            \end{itemize}
            \item pokud jsou ve druhé stopě samé terminály, porovnáme ji s první stopou. Slovo přijmeme nebo ho zamítneme.
        \end{itemize}
    \end{proof}
\end{theorem}
\begin{theorem}
    LBA přijímají pouze kontextové jazyky
    \begin{proof}
        \begin{itemize}
            \item potřebujeme převést LBA na monotónní gramatiku, t.j. grmaatika nesmí generovat nic navíc
            \item výpočet ukryjeme do 'dvoustupých' neterminálů
            \item generuj slovo ve tvaru $(a_0,[q_0,l,a_0]),(a_1,a_1),\dots,(a_n,[a_n,r])$
            \item simuluj práci LBA ve 'druhé' stopě (stejně jako u TM)
            \begin{itemize}
                \item pro $\delta(p,x) = (q,x',R): Px \rightarrow x'Q$
                \item pro $\delta(p,x) = (q,x',L): yPx \rightarrow Qyx'$
            \end{itemize}
            \item pokud je stav koncový smaž 'druhou' stopu
            \item speciálně je třeba ošetřit přijímaní prázdného slova (pokud LBA přijíma $\lambda$, přidáme speciální startovací pravidlo)
        \end{itemize}
    \end{proof}
\end{theorem}
\subsection{Rekurzivní jazyky}
\begin{definition}
    TM zastaví, pokud vstoupí do stavu $q$, s čteným symbolem $X$, a není instrukce pro tuto situaci, t. j. $\delta(q,X)$ není definováno.
\end{definition}
\begin{itemize}
    \item Předpokládáme, že v přijímacím stavu $q \in F$ TM zastaví,
    \item dokud nezastaví, nevíme jestli přijme nebo nepřijme slovo.
\end{itemize}
\begin{definition}
    Říkáme, že TM $M$ rozhoduje jazyk $L$, pokud $L = L(M)$ a pro každé $w \in \Sigma^*$ stroj nad $w$ zastaví.

    Jazyky rozhodnutelné TM nazýváme rekurzivní jazyky.
\end{definition}
\begin{theorem}
    Je-li L rekurzivní, je rekurzivní i $\overline{L}.$
\end{theorem}
\begin{theorem}
    Postova věta\\

    Jazyk $L$ je rekurzivní právě když $L$ i $\overline{L}$ (doplněk) jsou rekurzivně spočetné.
    \begin{proof}
        \begin{itemize}
            \item Máme $TM L = L(M_1) \& \overline{L} = L(M_2)$.
            \item pro dané slovo $w$ naraz simulujeme $M_1$ i $M_2$ (dvě pásky, stav se dvěma komponentami)
            \item Pokud jeden z $M_i$ přijme, $M$ zastaví a odpoví.
            \item Jazyky jsou komplementární, jeden z $M_i$ vždy zastaví, $L$ je rekurzivní.
        \end{itemize}
    \end{proof}
\end{theorem}
\subsection{Jazyk, který není rekurzivně spočetný}
Směřujeme k důkazu nerozhodnutelnosti jazyka dvojic $(M,w)$ takových, že:
\begin{itemize}
    \item $M$ je binárně kódovaný Turingův stroj s abecedou ${0,1}$,
    \item $w \in {0,1}^*$ a
    \item $M$ nepřijímá vstup $w$.
\end{itemize}
Postup:
\begin{itemize}
    \item Kódování TM binárním kódem pro libovolný počet stavů TM
    \item Kód TM vezmeme TM jako binární řetězec.
    \item Pokud kód nedáva smysl, reprezentuje TM bez transakcí. Tedy každý kód reprezentuje nějaký TM.
    \item Diagonální jazyk $L_d = \{w; \text{TM reprezentovaný jako } w \text{ takový, že nepřijímá }w\}$
    \item Neexistuje TM přijímající jazyk $L_d$. Spuštění takového stroje na vlastním kódu by vedlo k paradoxu. 
\end{itemize}
Jazyk $L_d$ není rekurzivně spočetný. Proto $\overline{L_d}$ není rekurzivní. Lze dokázat.
\subsection{Kódování}
\begin{itemize}
    \item Pro kódování $TM M = (Q,{0,1},\Gamma,\delta,q_1,B,{q_2})$ očíslujeme stavy, symboly a směry $L,R$.
    \item Předpokládejme:
    \begin{itemize}
        \item Počáteční stav je vždy $q_1$.
        \item Stav $q_2$ je vždy jediný koncový stav (nepotřebujeme víc, TM zastaví).
        \item První symbol je vždy 0, druhý 1, třetí B, prázdný symbol. OStatní symboly pásky očíslujeme libovolně.
        \item Směr $L$ je 1, směr $R$ je 2.
    \end{itemize}
    \item Jeden krok $\delta(q_i,X_j) = (q_k,X_l,D_m)$ kódujeme. $0^j10^j10^k10^l10^m$. Všechna $i,j,k,l,m \geq 1$, takže se dvě jedinčky za sebou nevyskytují.
    \item Celý TM se skládá z kódů všech přechodů v nějakém pořadí oddělených dvojicemi jedniček 11 - $C_1 11 C_2 11 \dots C_{n-1} 11 C_n$.
\end{itemize}
Budeme potřebovat uspořádat řetězce do posloupnosti:
\begin{itemize}
    \item Řetězce budeme uspořádávat podle délky, stejně dlouhé uspořádáme lexikograficky.
    \item První je $\lambda$, druhý 0, třetí 1, čtvrtý 11
    \item \textbf{TODO, ešte jedna vec tu bola}
\end{itemize}
\begin{definition}
    Diagonální jazyk $L_d$ je definovaný jako $L_d = \{w; \text{TM reprezentovaný jako } w \text{ takový, že nepřijímá }w\}$
\end{definition}
\begin{theorem}
    $L_d$ není rekurzivně spočetný jazyk, t. j. neexistuje TM přijímající $L_d$
    \begin{proof}
        \begin{itemize}
            \item Předpokládejme $L_d$e RE, $L_d = L(M_d)$ pro nějaký TM $M_d$.
            \item Jazyk je ${0,1}$. Přijímá TM $M_i$ vstuní slovo $w_j$?
            \item Alespoň jeden řetězec ho kóduje. řekněme, že $code(M_d) = w_d.$
            \item Je $w_D \in L_d$? 
            \begin{itemize}
                \item Pokud 'ano', na diagonále má být 0, t. j. $w_d \notin L(M_d) = L_d$, spor.
                \item Pokud 'ne', na diagonále má být 1, t. j. $w_d \in L(M_d) = L_d$, spor.
            \end{itemize}
        \end{itemize}
        Proto takový $M_d$ neexistuje. Tedy $L_d$ není rekuzrivně spočetný.
    \end{proof}
\end{theorem}
\begin{definition}
    Univerzální jazyk definujeme $L_u$ jakožto množinu binárních řetězců, které kódují pár $(M,w)$, kde $M$ je TM a $w \in L(M)$.

    TM který spracuje $L_u$ se nazývá Univerzální Turingův stroj.
\end{definition}
\begin{theorem}
    Existuje Turingův stroj U, pro který $L_u = L(U)$.
    \begin{proof}
        Popíšeme U jako vícepáskový Turingův stroj.
        \begin{itemize}
            \item Přechody M jsou napsány na první pásce spolu s řetězcem $w$.
            \item Na druhé pásce simulujeme výpočet M, používající formát
            \item \textbf{TODO}
        \end{itemize}
        Operace U jsou následující:
        \begin{itemize}
            \item Otestuj, zda je kód M legitimní; pokud ne, U zastav bez přijetí.
            \item Inicializuj druhou pásku kódovaným slovem w: 10 pro 0 ve w, 100 pro 1, blank jsou nechané prázdne a nahrazeny 1000 pouze v případě potřeby.
            \item Napíš 0, počáteční stav M, na třetí pásku. Posuň hlavu druhé pásky na první simulované políčko.
            \item Simuluj jednotlivé přechody M
            \begin{itemize}
                \item Najdi na první pásce správnou transakci $0^i10^j10^k10^l10^m$, $0^i$ na pásce 3, $0^j$ na pásce 2.
                \item Změň obsah pásky 3 na $0^k$.
                \item Nahraď $0^j$ na 2. pásce řetězcem $O^l$. Použij čtvrtou 'scratch tape' pro správné mezery.
                \item Posuň hlavu 2. pásky na pozici vedle 1 vlevo nebo vpravo, podle pohybu m. 
            \end{itemize}
            \item pokud \textbf{TODO}
        \end{itemize}
    \end{proof}
\end{theorem}
\end{document}